\documentclass[12pt,titlepage,draft]{report}
%Gummi|065|=)
\usepackage[utf8]{inputenc}
\usepackage[slovene]{babel}
\usepackage{graphicx}
\usepackage{setspace}
\usepackage{mathptmx}
\usepackage{vegovatitle}
\usepackage[
        a4paper,% other options: a3paper, a5paper, etc
        left=3cm,
        right=2.5cm,
        top=2.5cm,
        bottom=2.5cm,
        % use vmargin=2cm to make vertical margins equal to 2cm.
        % us  hmargin=3cm to make horizontal margins equal to 3cm.
        % use margin=3cm to make all margins  equal to 3cm.
]{geometry}
\linespread{1.5}
\renewcommand\thesection{\arabic{section}}

\title{KALKULATOR}
\project{Strokovno poročilo}
\author{Rok Strah, R 4. C}
\mentor{Darjan Toth, prof.}

\begin{document}

\maketitle

\section*{Povzetek}
\section*{Abstract}
\thispagestyle{empty}


{\begin{spacing}{1.15}
\tableofcontents
\end{spacing}}
\thispagestyle{empty}
\clearpage
\setcounter{page}{1}

\section{\textbf{UVOD}}


\section{\textbf{METODOLOGIJA}}
\subsection{OKOLJE IN JEZIK}
Za projekt sem uporabil programski jezik C++, razvojno orodje Qt Creator in nekaj knjižnic.

\subsection{FUNKCIONALNOST}
Program dovoljuje vnos enacbe.
Po končanem računanju lahko slikovni prikaz enacbe z rešitvijo prekopiramo.
Na voljo je tudi kopiranje rezultata samega.


\section{\textbf{REZULTATI IN UGOTOVITVE}}
Projekt me je naučil dela s knjižnicami.

\subsection{NAVODILA ZA UPORABO}


\section{\textbf{VIRI IN LITERATURA}}
{qt docs}\\
{tex.stackexchange}\\
{exprtk page}\\



\end{document}


























































