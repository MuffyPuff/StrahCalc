
C++ Mathematical Expression Toolkit Library Documentation

Section 00 - Introduction
Section 01 - Capabilities
Section 02 - Example Expressions
Section 03 - Copyright Notice
Section 04 - Downloads \& Updates
Section 05 - Installation
Section 06 - Compilation
Section 07 - Compiler Compatibility
Section 08 - Built-In Operations \& Functions
Section 09 - Fundamental Types
Section 10 - Components
Section 11 - Compilation Options
Section 12 - Expression Structures
Section 13 - Variable, Vector \& String Definition
Section 14 - Vector Processing
Section 15 - User Defined Functions
Section 16 - Expression Dependents
Section 17 - Hierarchies Of Symbol Tables
Section 18 - Unknown Unknowns
Section 19 - Enabling \& Disabling Features
Section 20 - Expression Return Values
Section 21 - Compilation Errors
Section 22 - Runtime Library Packages
Section 23 - Helpers \& Utils
Section 24 - Benchmarking
Section 25 - Exprtk Notes
Section 26 - Simple Exprtk Example
Section 27 - Build Options
Section 28 - Files
Section 29 - Language Structure


[SECTION 00 - INTRODUCTION]
The C++ Mathematical Expression  Toolkit Library (ExprTk) is  a simple
to  use,   easy  to   integrate  and   extremely  efficient   run-time
mathematical  expression parsing  and evaluation  engine. The  parsing
engine  supports numerous  forms  of  functional and  logic processing
semantics and is easily extensible.

~~~~~~~~~~~~~~~~~~~~~~~~~~~~~~~~~~~~~~~~~~~~~~~~~~~~~~~~~~

[SECTION 01 - CAPABILITIES]
The  ExprTk expression  evaluator supports  the following  fundamental
arithmetic operations, functions and processes:

(00) Types:           Scalar, Vector, String

(01) Basic operators: +, -, *, /, \%, \^\\

(02) Assignment:      :=, +=, -=, *=, /=, \%=

(03) Equalities \&
Inequalities:    =, ==, <>, !=, <, <=, >, >=

(04) Logic operators: and, mand, mor, nand, nor, not, or, shl, shr,
xnor, xor, true, false

(05) Functions:       abs, avg, ceil, clamp, equal, erf, erfc,  exp,
expm1, floor, frac,  log, log10, log1p,  log2,
logn,  max,  min,  mul,  ncdf,  nequal,  root,
round, roundn, sgn, sqrt, sum, swap, trunc

(06) Trigonometry:    acos, acosh, asin, asinh, atan, atanh,  atan2,
cos,  cosh, cot,  csc, sec,  sin, sinc,  sinh,
tan, tanh, hypot, rad2deg, deg2grad,  deg2rad,
grad2deg

(07) Control
structures:      if-then-else, ternary conditional, switch-case,
return-statement

(08) Loop statements: while, for, repeat-until, break, continue

(09) String
processing:      in, like, ilike, concatenation

(10) Optimisations:   constant-folding, simple strength reduction and
dead code elimination

(11) Calculus:        numerical integration and differentiation

~~~~~~~~~~~~~~~~~~~~~~~~~~~~~~~~~~~~~~~~~~~~~~~~~~~~~~~~~~

[SECTION 02 - EXAMPLE EXPRESSIONS]
The following is  a short listing  of infix format  based mathematical
expressions that can be parsed and evaluated using the ExprTk library.

(01) sqrt(1 - (3 / x\^2))
(02) clamp(-1, sin(2 * pi * x) + cos(y / 2 * pi), +1)
(03) sin(2.34e-3 * x)
(04) if(((x[2] + 2) == 3) and ((y + 5) <= 9),1 + w, 2 / z)
(05) inrange(-2,m,+2) == if(({-2 <= m} and [m <= +2]),1,0)
(06) ({1/1}*[1/2]+(1/3))-{1/4}\^[1/5]+(1/6)-({1/7}+[1/8]*(1/9))
(07) a * exp(2.2 / 3.3 * t) + c
(08) z := x + sin(2.567 * pi / y)
(09) u := 2.123 * {pi * z} / (w := x + cos(y / pi))
(10) 2x + 3y + 4z + 5w == 2 * x + 3 * y + 4 * z + 5 * w
(11) 3(x + y) / 2.9 + 1.234e+12 == 3 * (x + y) / 2.9 + 1.234e+12
(12) (x + y)3.3 + 1 / 4.5 == [x + y] * 3.3 + 1 / 4.5
(13) (x + y[i])z + 1.1 / 2.7 == (x + y[i]) * z + 1.1 / 2.7
(14) (sin(x / pi) cos(2y) + 1) == (sin(x / pi) * cos(2 * y) + 1)
(15) 75x\^17 + 25.1x\^5 - 35x\^4 - 15.2x\^3 + 40x\^2 - 15.3x + 1
(16) (avg(x,y) <= x + y ? x - y : x * y) + 2.345 * pi / x
(17) while (x <= 100) { x -= 1; }
(18) x <= 'abc123' and (y in 'AString') or ('1x2y3z' != z)
(19) ((x + 'abc') like '*123*') or ('a123b' ilike y)
(20) sgn(+1.2\^3.4z / -5.6y) <= {-7.8\^9 / -10.11x }

~~~~~~~~~~~~~~~~~~~~~~~~~~~~~~~~~~~~~~~~~~~~~~~~~~~~~~~~~~

[SECTION 03 - COPYRIGHT NOTICE]
Free  use  of  the  C++  Mathematical  Expression  Toolkit  Library is
permitted under the guidelines and in accordance with the most current
version of the MIT License.

http://www.opensource.org/licenses/MIT

~~~~~~~~~~~~~~~~~~~~~~~~~~~~~~~~~~~~~~~~~~~~~~~~~~~~~~~~~~

[SECTION 04 - DOWNLOADS \& UPDATES]
The most  recent version  of the C++ Mathematical  Expression  Toolkit
Library including all updates and tests can be found at the  following
locations:

(a) Download:   http://www.partow.net/programming/exprtk/index.html
(b) Repository: https://github.com/ArashPartow/exprtk
https://github.com/ArashPartow/exprtk-extras

~~~~~~~~~~~~~~~~~~~~~~~~~~~~~~~~~~~~~~~~~~~~~~~~~~~~~~~~~~

[SECTION 05 - INSTALLATION]
The header  file exprtk.hpp  should be  placed in a project or  system
include path (e.g: /usr/include/).

~~~~~~~~~~~~~~~~~~~~~~~~~~~~~~~~~~~~~~~~~~~~~~~~~~~~~~~~~~

[SECTION 06 - COMPILATION]
(a) For a complete build: make clean all
(b) For a PGO build: make clean pgo
(c) To strip executables: make strip\_bin
(d) Execute valgrind check: make valgrind\_check

~~~~~~~~~~~~~~~~~~~~~~~~~~~~~~~~~~~~~~~~~~~~~~~~~~~~~~~~~~

[SECTION 07 - COMPILER COMPATIBILITY]
ExprTk has been built error and warning free using the following set
of C++ compilers:

(*) GNU Compiler Collection (3.5+)
(*) Intel C++ Compiler (8.x+)
(*) Clang/LLVM (1.1+)
(*) PGI C++ (10.x+)
(*) Microsoft Visual Studio C++ Compiler (8.1+)
(*) IBM XL C/C++ (9.x+)
(*) C++ Builder (XE4+)

~~~~~~~~~~~~~~~~~~~~~~~~~~~~~~~~~~~~~~~~~~~~~~~~~~~~~~~~~~

[SECTION 08 - BUILT-IN OPERATIONS \& FUNCTIONS]

(0) Arithmetic \& Assignment Operators
+----------+---------------------------------------------------------+
| OPERATOR | DEFINITION                                              |
+----------+---------------------------------------------------------+
|  +       | Addition between x and y.  (eg: x + y)                  |
+----------+---------------------------------------------------------+
|  -       | Subtraction between x and y.  (eg: x - y)               |
+----------+---------------------------------------------------------+
|  *       | Multiplication between x and y.  (eg: x * y)            |
+----------+---------------------------------------------------------+
|  /       | Division between x and y.  (eg: x / y)                  |
+----------+---------------------------------------------------------+
|  \%       | Modulus of x with respect to y.  (eg: x \% y)            |
+----------+---------------------------------------------------------+
|  \^       | x to the power of y.  (eg: x \^ y)                       |
+----------+---------------------------------------------------------+
|  :=      | Assign the value of x to y. Where y is either a variable|
|          | or vector type.  (eg: y := x)                           |
+----------+---------------------------------------------------------+
|  +=      | Increment x by the value of the expression on the right |
|          | hand side. Where x is either a variable or vector type. |
|          | (eg: x += abs(y - z))                                   |
+----------+---------------------------------------------------------+
|  -=      | Decrement x by the value of the expression on the right |
|          | hand side. Where x is either a variable or vector type. |
|          | (eg: x[i] -= abs(y + z))                                |
+----------+---------------------------------------------------------+
|  *=      | Assign the multiplication of x by the value of the      |
|          | expression on the righthand side to x. Where x is either|
|          | a variable or vector type.                              |
|          | (eg: x *= abs(y / z))                                   |
+----------+---------------------------------------------------------+
|  /=      | Assign the division of x by the value of the expression |
|          | on the right-hand side to x. Where x is either a        |
|          | variable or vector type.  (eg: x[i + j] /= abs(y * z))  |
+----------+---------------------------------------------------------+
|  \%=      | Assign x modulo the value of the expression on the right|
|          | hand side to x. Where x is either a variable or vector  |
|          | type.  (eg: x[2] \%= y \^ 2)                              |
+----------+---------------------------------------------------------+

(1) Equalities \& Inequalities
+----------+---------------------------------------------------------+
| OPERATOR | DEFINITION                                              |
+----------+---------------------------------------------------------+
| == or =  | True only if x is strictly equal to y. (eg: x == y)     |
+----------+---------------------------------------------------------+
| <> or != | True only if x does not equal y. (eg: x <> y or x != y) |
+----------+---------------------------------------------------------+
|  <       | True only if x is less than y. (eg: x < y)              |
+----------+---------------------------------------------------------+
|  <=      | True only if x is less than or equal to y. (eg: x <= y) |
+----------+---------------------------------------------------------+
|  >       | True only if x is greater than y. (eg: x > y)           |
+----------+---------------------------------------------------------+
|  >=      | True only if x greater than or equal to y. (eg: x >= y) |
+----------+---------------------------------------------------------+

(2) Boolean Operations
+----------+---------------------------------------------------------+
| OPERATOR | DEFINITION                                              |
+----------+---------------------------------------------------------+
| true     | True state or any value other than zero (typically 1).  |
+----------+---------------------------------------------------------+
| false    | False state, value of exactly zero.                     |
+----------+---------------------------------------------------------+
| and      | Logical AND, True only if x and y are both true.        |
|          | (eg: x and y)                                           |
+----------+---------------------------------------------------------+
| mand     | Multi-input logical AND, True only if all inputs are    |
|          | true. Left to right short-circuiting of expressions.    |
|          | (eg: mand(x > y, z < w, u or v, w and x))               |
+----------+---------------------------------------------------------+
| mor      | Multi-input logical OR, True if at least one of the     |
|          | inputs are true. Left to right short-circuiting of      |
|          | expressions.  (eg: mor(x > y, z < w, u or v, w and x))  |
+----------+---------------------------------------------------------+
| nand     | Logical NAND, True only if either x or y is false.      |
|          | (eg: x nand y)                                          |
+----------+---------------------------------------------------------+
| nor      | Logical NOR, True only if the result of x or y is false |
|          | (eg: x nor y)                                           |
+----------+---------------------------------------------------------+
| not      | Logical NOT, Negate the logical sense of the input.     |
|          | (eg: not(x and y) == x nand y)                          |
+----------+---------------------------------------------------------+
| or       | Logical OR, True if either x or y is true. (eg: x or y) |
+----------+---------------------------------------------------------+
| xor      | Logical XOR, True only if the logical states of x and y |
|          | differ.  (eg: x xor y)                                  |
+----------+---------------------------------------------------------+
| xnor     | Logical XNOR, True iff the biconditional of x and y is  |
|          | satisfied.  (eg: x xnor y)                              |
+----------+---------------------------------------------------------+
| \&        | Similar to AND but with left to right expression short  |
|          | circuiting optimisation.  (eg: (x \& y) == (y and x))    |
+----------+---------------------------------------------------------+
| |        | Similar to OR but with left to right expression short   |
|          | circuiting optimisation.  (eg: (x | y) == (y or x))     |
+----------+---------------------------------------------------------+

(3) General Purpose Functions
+----------+---------------------------------------------------------+
| FUNCTION | DEFINITION                                              |
+----------+---------------------------------------------------------+
| abs      | Absolute value of x.  (eg: abs(x))                      |
+----------+---------------------------------------------------------+
| avg      | Average of all the inputs.                              |
|          | (eg: avg(x,y,z,w,u,v) == (x + y + z + w + u + v) / 6)   |
+----------+---------------------------------------------------------+
| ceil     | Smallest integer that is greater than or equal to x.    |
+----------+---------------------------------------------------------+
| clamp    | Clamp x in range between r0 and r1, where r0 < r1.      |
|          | (eg: clamp(r0,x,r1))                                    |
+----------+---------------------------------------------------------+
| equal    | Equality test between x and y using normalised epsilon  |
+----------+---------------------------------------------------------+
| erf      | Error function of x.  (eg: erf(x))                      |
+----------+---------------------------------------------------------+
| erfc     | Complimentary error function of x.  (eg: erfc(x))       |
+----------+---------------------------------------------------------+
| exp      | e to the power of x.  (eg: exp(x))                      |
+----------+---------------------------------------------------------+
| expm1    | e to the power of x minus 1, where x is very small.     |
|          | (eg: expm1(x))                                          |
+----------+---------------------------------------------------------+
| floor    | Largest integer that is less than or equal to x.        |
|          | (eg: floor(x))                                          |
+----------+---------------------------------------------------------+
| frac     | Fractional portion of x.  (eg: frac(x))                 |
+----------+---------------------------------------------------------+
| hypot    | Hypotenuse of x and y (eg: hypot(x,y) = sqrt(x*x + y*y))|
+----------+---------------------------------------------------------+
| iclamp   | Inverse-clamp x outside of the range r0 and r1. Where   |
|          | r0 < r1. If x is within the range it will snap to the   |
|          | closest bound. (eg: iclamp(r0,x,r1)                     |
+----------+---------------------------------------------------------+
| inrange  | In-range returns 'true' when x is within the range r0   |
|          | and r1. Where r0 < r1.  (eg: inrange(r0,x,r1)           |
+----------+---------------------------------------------------------+
| log      | Natural logarithm of x.  (eg: log(x))                   |
+----------+---------------------------------------------------------+
| log10    | Base 10 logarithm of x.  (eg: log10(x))                 |
+----------+---------------------------------------------------------+
| log1p    | Natural logarithm of 1 + x, where x is very small.      |
|          | (eg: log1p(x))                                          |
+----------+---------------------------------------------------------+
| log2     | Base 2 logarithm of x.  (eg: log2(x))                   |
+----------+---------------------------------------------------------+
| logn     | Base N logarithm of x. where n is a positive integer.   |
|          | (eg: logn(x,8))                                         |
+----------+---------------------------------------------------------+
| max      | Largest value of all the inputs. (eg: max(x,y,z,w,u,v)) |
+----------+---------------------------------------------------------+
| min      | Smallest value of all the inputs. (eg: min(x,y,z,w,u))  |
+----------+---------------------------------------------------------+
| mul      | Product of all the inputs.                              |
|          | (eg: mul(x,y,z,w,u,v,t) == (x * y * z * w * u * v * t)) |
+----------+---------------------------------------------------------+
| ncdf     | Normal cumulative distribution function.  (eg: ncdf(x)) |
+----------+---------------------------------------------------------+
| nequal   | Not-equal test between x and y using normalised epsilon |
+----------+---------------------------------------------------------+
| pow      | x to the power of y.  (eg: pow(x,y) == x \^ y)           |
+----------+---------------------------------------------------------+
| root     | Nth-Root of x. where n is a positive integer.           |
|          | (eg: root(x,3) == x\^(1/3))                              |
+----------+---------------------------------------------------------+
| round    | Round x to the nearest integer.  (eg: round(x))         |
+----------+---------------------------------------------------------+
| roundn   | Round x to n decimal places  (eg: roundn(x,3))          |
|          | where n > 0 and is an integer.                          |
|          | (eg: roundn(1.2345678,4) == 1.2346)                     |
+----------+---------------------------------------------------------+
| sgn      | Sign of x, -1 where x < 0, +1 where x > 0, else zero.   |
|          | (eg: sgn(x))                                            |
+----------+---------------------------------------------------------+
| sqrt     | Square root of x, where x >= 0.  (eg: sqrt(x))          |
+----------+---------------------------------------------------------+
| sum      | Sum of all the inputs.                                  |
|          | (eg: sum(x,y,z,w,u,v,t) == (x + y + z + w + u + v + t)) |
+----------+---------------------------------------------------------+
| swap     | Swap the values of the variables x and y and return the |
| <=>      | current value of y.  (eg: swap(x,y) or x <=> y)         |
+----------+---------------------------------------------------------+
| trunc    | Integer portion of x.  (eg: trunc(x))                   |
+----------+---------------------------------------------------------+

(4) Trigonometry Functions
+----------+---------------------------------------------------------+
| FUNCTION | DEFINITION                                              |
+----------+---------------------------------------------------------+
| acos     | Arc cosine of x expressed in radians. Interval [-1,+1]  |
|          | (eg: acos(x))                                           |
+----------+---------------------------------------------------------+
| acosh    | Inverse hyperbolic cosine of x expressed in radians.    |
|          | (eg: acosh(x))                                          |
+----------+---------------------------------------------------------+
| asin     | Arc sine of x expressed in radians. Interval [-1,+1]    |
|          | (eg: asin(x))                                           |
+----------+---------------------------------------------------------+
| asinh    | Inverse hyperbolic sine of x expressed in radians.      |
|          | (eg: asinh(x))                                          |
+----------+---------------------------------------------------------+
| atan     | Arc tangent of x expressed in radians. Interval [-1,+1] |
|          | (eg: atan(x))                                           |
+----------+---------------------------------------------------------+
| atan2    | Arc tangent of (x / y) expressed in radians. [-pi,+pi]  |
|          | eg: atan2(x,y)                                          |
+----------+---------------------------------------------------------+
| atanh    | Inverse hyperbolic tangent of x expressed in radians.   |
|          | (eg: atanh(x))                                          |
+----------+---------------------------------------------------------+
| cos      | Cosine of x.  (eg: cos(x))                              |
+----------+---------------------------------------------------------+
| cosh     | Hyperbolic cosine of x.  (eg: cosh(x))                  |
+----------+---------------------------------------------------------+
| cot      | Cotangent of x.  (eg: cot(x))                           |
+----------+---------------------------------------------------------+
| csc      | Cosecant of x.  (eg: csc(x))                            |
+----------+---------------------------------------------------------+
| sec      | Secant of x.  (eg: sec(x))                              |
+----------+---------------------------------------------------------+
| sin      | Sine of x.  (eg: sin(x))                                |
+----------+---------------------------------------------------------+
| sinc     | Sine cardinal of x.  (eg: sinc(x))                      |
+----------+---------------------------------------------------------+
| sinh     | Hyperbolic sine of x.  (eg: sinh(x))                    |
+----------+---------------------------------------------------------+
| tan      | Tangent of x.  (eg: tan(x))                             |
+----------+---------------------------------------------------------+
| tanh     | Hyperbolic tangent of x.  (eg: tanh(x))                 |
+----------+---------------------------------------------------------+
| deg2rad  | Convert x from degrees to radians.  (eg: deg2rad(x))    |
+----------+---------------------------------------------------------+
| deg2grad | Convert x from degrees to gradians.  (eg: deg2grad(x))  |
+----------+---------------------------------------------------------+
| rad2deg  | Convert x from radians to degrees.  (eg: rad2deg(x))    |
+----------+---------------------------------------------------------+
| grad2deg | Convert x from gradians to degrees.  (eg: grad2deg(x))  |
+----------+---------------------------------------------------------+

(5) String Processing
+----------+---------------------------------------------------------+
| FUNCTION | DEFINITION                                              |
+----------+---------------------------------------------------------+
|  = , ==  | All common equality/inequality operators are applicable |
|  !=, <>  | to strings and are applied in a case sensitive manner.  |
|  <=, >=  | In the following example x, y and z are of type string. |
|  < , >   | (eg: not((x <= 'AbC') and ('1x2y3z' <> y)) or (z == x)  |
+----------+---------------------------------------------------------+
| in       | True only if x is a substring of y.                     |
|          | (eg: x in y or 'abc' in 'abcdefgh')                     |
+----------+---------------------------------------------------------+
| like     | True only if the string x matches the pattern y.        |
|          | Available wildcard characters are '*' and '?' denoting  |
|          | zero or more and zero or one matches respectively.      |
|          | (eg: x like y or 'abcdefgh' like 'a?d*h')               |
+----------+---------------------------------------------------------+
| ilike    | True only if the string x matches the pattern y in a    |
|          | case insensitive manner. Available wildcard characters  |
|          | are '*' and '?' denoting zero or more and zero or one   |
|          | matches respectively.                                   |
|          | (eg: x ilike y or 'a1B2c3D4e5F6g7H' ilike 'a?d*h')      |
+----------+---------------------------------------------------------+
| [r0:r1]  | The closed interval [r0,r1] of the specified string.    |
|          | eg: Given a string x with a value of 'abcdefgh' then:   |
|          | 1. x[1:4] == 'bcde'                                     |
|          | 2. x[ :5] == x[:5] == 'abcdef'                          |
|          | 3. x[3: ] == x[3:] =='cdefgh'                           |
|          | 4. x[ : ] == x[:] == 'abcdefgh'                         |
|          | 5. x[4/2:3+2] == x[2:5] == 'cdef'                       |
|          |                                                         |
|          | Note: Both r0 and r1 are assumed to be integers, where  |
|          | r0 <= r1. They may also be the result of an expression, |
|          | in the event they have fractional components truncation |
|          | will be performed. (eg: 1.67 --> 1)                     |
+----------+---------------------------------------------------------+
|  :=      | Assign the value of x to y. Where y is a mutable string |
|          | or string range and x is either a string or a string    |
|          | range. eg:                                              |
|          | 1. y := x                                               |
|          | 2. y := 'abc'                                           |
|          | 3. y := x[:i + j]                                       |
|          | 4. y := '0123456789'[2:7]                               |
|          | 5. y := '0123456789'[2i + 1:7]                          |
|          | 6. y := (x := '0123456789'[2:7])                        |
|          | 7. y[i:j] := x                                          |
|          | 8. y[i:j] := (x + 'abcdefg'[8 / 4:5])[m:n]              |
|          |                                                         |
|          | Note: For options 7 and 8 the shorter of the two ranges |
|          | will denote the number characters that are to be copied.|
+----------+---------------------------------------------------------+
|  +       | Concatenation of x and y. Where x and y are strings or  |
|          | string ranges. eg                                       |
|          | 1. x + y                                                |
|          | 2. x + 'abc'                                            |
|          | 3. x + y[:i + j]                                        |
|          | 4. x[i:j] + y[2:3] + '0123456789'[2:7]                  |
|          | 5. 'abc' + x + y                                        |
|          | 6. 'abc' + '1234567'                                    |
|          | 7. (x + 'a1B2c3D4' + y)[i:2j]                           |
+----------+---------------------------------------------------------+
|  +=      | Append to x the value of y. Where x is a mutable string |
|          | and y is either a string or a string range. eg:         |
|          | 1. x += y                                               |
|          | 2. x += 'abc'                                           |
|          | 3. x += y[:i + j] + 'abc'                               |
|          | 4. x += '0123456789'[2:7]                               |
+----------+---------------------------------------------------------+
| <=>      | Swap the values of x and y. Where x and y are mutable   |
|          | strings.  (eg: x <=> y)                                 |
+----------+---------------------------------------------------------+
| []       | The string size operator returns the size of the string |
|          | being actioned.                                         |
|          | eg:                                                     |
|          | 1. 'abc'[] == 3                                         |
|          | 2. var max\_str\_length := max(s0[],s1[],s2[],s3[])       |
|          | 3. ('abc' + 'xyz')[] == 6                               |
|          | 4. (('abc' + 'xyz')[1:4])[] == 4                        |
+----------+---------------------------------------------------------+

(6) Control Structures
+----------+---------------------------------------------------------+
|STRUCTURE | DEFINITION                                              |
+----------+---------------------------------------------------------+
| if       | If x is true then return y else return z.               |
|          | eg:                                                     |
|          | 1. if (x, y, z)                                         |
|          | 2. if ((x + 1) > 2y, z + 1, w / v)                      |
|          | 3. if (x > y) z;                                        |
|          | 4. if (x <= 2*y) { z + w };                             |
+----------+---------------------------------------------------------+
| if-else  | The if-else/else-if statement. Subject to the condition |
|          | branch the statement will return either the value of the|
|          | consequent or the alternative branch.                   |
|          | eg:                                                     |
|          | 1. if (x > y) z; else w;                                |
|          | 2. if (x > y) z; else if (w != u) v;                    |
|          | 3. if (x < y) { z; w + 1; } else u;                     |
|          | 4. if ((x != y) and (z > w))                            |
|          |    {                                                    |
|          |      y := sin(x) / u;                                   |
|          |      z := w + 1;                                        |
|          |    }                                                    |
|          |    else if (x > (z + 1))                                |
|          |    {                                                    |
|          |      w := abs (x - y) + z;                              |
|          |      u := (x + 1) > 2y ? 2u : 3u;                       |
|          |    }                                                    |
+----------+---------------------------------------------------------+
| switch   | The first true case condition that is encountered will  |
|          | determine the result of the switch. If none of the case |
|          | conditions hold true, the default action is assumed as  |
|          | the final return value. This is sometimes also known as |
|          | a multi-way branch mechanism.                           |
|          | eg:                                                     |
|          | switch                                                  |
|          | {                                                       |
|          |   case x > (y + z) : 2 * x / abs(y - z);                |
|          |   case x < 3       : sin(x + y);                        |
|          |   default          : 1 + x;                             |
|          | }                                                       |
+----------+---------------------------------------------------------+
| while    | The structure will repeatedly evaluate the internal     |
|          | statement(s) 'while' the condition is true. The final   |
|          | statement in the final iteration will be used as the    |
|          | return value of the loop.                               |
|          | eg:                                                     |
|          | while ((x -= 1) > 0)                                    |
|          | {                                                       |
|          |   y := x + z;                                           |
|          |   w := u + y;                                           |
|          | }                                                       |
+----------+---------------------------------------------------------+
| repeat/  | The structure will repeatedly evaluate the internal     |
| until    | statement(s) 'until' the condition is true. The final   |
|          | statement in the final iteration will be used as the    |
|          | return value of the loop.                               |
|          | eg:                                                     |
|          | repeat                                                  |
|          |   y := x + z;                                           |
|          |   w := u + y;                                           |
|          | until ((x += 1) > 100)                                  |
+----------+---------------------------------------------------------+
| for      | The structure will repeatedly evaluate the internal     |
|          | statement(s) while the condition is true. On each loop  |
|          | iteration, an 'incrementing' expression is evaluated.   |
|          | The conditional is mandatory whereas the initialiser    |
|          | and incrementing expressions are optional.              |
|          | eg:                                                     |
|          | for (var x := 0; (x < n) and (x != y); x += 1)          |
|          | {                                                       |
|          |   y := y + x / 2 - z;                                   |
|          |   w := u + y;                                           |
|          | }                                                       |
+----------+---------------------------------------------------------+
| break    | Break terminates the execution of the nearest enclosed  |
| break[]  | loop, allowing for the execution to continue on external|
|          | to the loop. The default break statement will set the   |
|          | return value of the loop to NaN, where as the return    |
|          | based form will set the value to that of the break      |
|          | expression.                                             |
|          | eg:                                                     |
|          | while ((i += 1) < 10)                                   |
|          | {                                                       |
|          |   if (i < 5)                                            |
|          |     j -= i + 2;                                         |
|          |   else if (i \% 2 == 0)                                  |
|          |     break;                                              |
|          |   else                                                  |
|          |     break[2i + 3];                                      |
|          | }                                                       |
+----------+---------------------------------------------------------+
| continue | Continue results in the remaining portion of the nearest|
|          | enclosing loop body to be skipped.                      |
|          | eg:                                                     |
|          | for (var i := 0; i < 10; i += 1)                        |
|          | {                                                       |
|          |   if (i < 5)                                            |
|          |     continue;                                           |
|          |   j -= i + 2;                                           |
|          | }                                                       |
+----------+---------------------------------------------------------+
| return   | Return immediately from within the current expression.  |
|          | With the option of passing back a variable number of    |
|          | values (scalar, vector or string). eg:                  |
|          | 1. return [1];                                          |
|          | 2. return [x, 'abx'];                                   |
|          | 3. return [x, x + y,'abx'];                             |
|          | 4. return [];                                           |
|          | 5. if (x < y)                                           |
|          |     return [x, x - y, 'result-set1', 123.456];          |
|          |    else                                                 |
|          |     return [y, x + y, 'result-set2'];                   |
+----------+---------------------------------------------------------+
| ?:       | Ternary conditional statement, similar to that of the   |
|          | above denoted if-statement.                             |
|          | eg:                                                     |
|          | 1. x ? y : z                                            |
|          | 2. x + 1 > 2y ? z + 1 : (w / v)                         |
|          | 3. min(x,y) > z ? (x < y + 1) ? x : y : (w * v)         |
+----------+---------------------------------------------------------+
| ~        | Evaluate each sub-expression, then return as the result |
|          | the value of the last sub-expression. This is sometimes |
|          | known as multiple sequence point evaluation.            |
|          | eg:                                                     |
|          | ~(i := x + 1, j := y / z, k := sin(w/u)) == (sin(w/u))) |
|          | ~{i := x + 1; j := y / z; k := sin(w/u)} == (sin(w/u))) |
+----------+---------------------------------------------------------+
| [*]      | Evaluate any consequent for which its case statement is |
|          | true. The return value will be either zero or the result|
|          | of the last consequent to have been evaluated.          |
|          | eg:                                                     |
|          | [*]                                                     |
|          | {                                                       |
|          |   case (x + 1) > (y - 2)    : x := z / 2 + sin(y / pi); |
|          |   case (x + 2) < abs(y + 3) : w / 4 + min(5y,9);        |
|          |   case (x + 3) == (y * 4)   : y := abs(z / 6) + 7y;     |
|          | }                                                       |
+----------+---------------------------------------------------------+
| []       | The vector size operator returns the size of the vector |
|          | being actioned.                                         |
|          | eg:                                                     |
|          | 1. v[]                                                  |
|          | 2. max\_size := max(v0[],v1[],v2[],v3[])                 |
+----------+---------------------------------------------------------+

Note: In  the  tables  above, the  symbols x, y, z, w, u  and v  where
appropriate may represent any of one the following:

1. Literal numeric/string value
2. A variable
3. A vector element
4. A vector
5. A string
6. An expression comprised of [1], [2] or [3] (eg: 2 + x / vec[3])

~~~~~~~~~~~~~~~~~~~~~~~~~~~~~~~~~~~~~~~~~~~~~~~~~~~~~~~~~~

[SECTION 09 - FUNDAMENTAL TYPES]
ExprTk supports three fundamental types which can be used freely in
expressions. The types are as follows:

(1) Scalar
(2) Vector
(3) String


(1) Scalar Type
The scalar type  is a singular  numeric value. The  underlying type is
that used  to specialise  the ExprTk  components (float,  double, long
double, MPFR et al).


(2) Vector Type
The vector type is a fixed size sequence of contiguous scalar  values.
A  vector  can be  indexed  resulting in  a  scalar value.  Operations
between a vector and scalar will result in a vector with a size  equal
to that  of the  original vector,  whereas operations  between vectors
will result in a  vector of size equal  to that of the  smaller of the
two. In both mentioned cases, the operations will occur element-wise.


(3) String Type
The string type is a variable length sequence of 8-bit chars.  Strings
can be  assigned and  concatenated to  one another,  they can  also be
manipulated via sub-ranges using the range definition syntax.  Strings
however can not interact with scalar or vector types.

~~~~~~~~~~~~~~~~~~~~~~~~~~~~~~~~~~~~~~~~~~~~~~~~~~~~~~~~~~

[SECTION 10 - COMPONENTS]
There are three primary components, that are specialised upon a  given
numeric type, which make up the core of ExprTk. The components are  as
follows:

(1) Symbol Table  exprtk::symbol\_table<NumericType>
(2) Expression    exprtk::expression<NumericType>
(3) Parser        exprtk::parser<NumericType>


(1) Symbol Table
A structure that is used  to store references to variables,  constants
and functions that are to  be used within expressions. Furthermore  in
the context  of composited  recursive functions  the symbol  table can
also be thought of as a simple representation of a stack specific  for
the expression(s) that reference it. The following is a list of the
types a symbol table can handle:

(a) Numeric variables
(b) Numeric constants
(c) Numeric vector elements
(d) String variables
(e) String constants
(f) Functions
(g) Vararg functions

During the compilation  process if an  expression is found  to require
any  of  the  elements   noted  above,  the  expression's   associated
symbol\_table  will  be  queried  for  the  element  and  if  present a
reference to the element will be embedded within the expression's AST.
This allows for the original  element to be modified independently  of
the  expression  instance  and  to also  allow  the  expression  to be
evaluated using the current value of the element.

The  example  below demonstrates  the  relationship between variables,
symbol\_table and expression. Note  the variables are modified  as they
normally would in a program, and when the expression is  evaluated the
current values assigned to the variables will be used.

typedef exprtk::symbol\_table<double> symbol\_table\_t;
typedef exprtk::expression<double>     expression\_t;
typedef exprtk::parser<double>             parser\_t;

symbol\_table\_t symbol\_table;
expression\_t   expression;
parser\_t       parser;

double x = 0;
double y = 0;

std::string expression\_string = "x * y + 3";
symbol\_table.add\_variable("x",x);
symbol\_table.add\_variable("y",y);

expression.register\_symbol\_table(symbol\_table);

parser.compile(expression\_string,expression);

x = 1.0;
y = 2.0;
expression.value(); // 1 * 2 + 3

x = 3.7;
expression.value(); // 3.7 * 2 + 3

y = -9.0;
expression.value(); // 3.7 * -9 + 3

// 'x * -9 + 3' for x in range of [0,100) in steps of 0.0001
for (x = 0.0; x < 100.0; x += 0.0001)
{
expression.value(); // x * -9 + 3
}


Note: It is possible to register multiple symbol\_tables with a  single
expression  object. In  the event  an expression  has multiple  symbol
tables,  and  where  there  exists  conflicts  between  symbols,   the
compilation stage  will resolve  the conflicts  based on  the order of
registration  of  the  symbol\_tables to  the  expression.  For a  more
expansive  discussion  please  review section  [17  -  Hierarchies  Of
Symbol Tables]

typedef exprtk::symbol\_table<double> symbol\_table\_t;
typedef exprtk::expression<double>     expression\_t;
typedef exprtk::parser<double>             parser\_t;

symbol\_table\_t symbol\_table0;
symbol\_table\_t symbol\_table1;

expression\_t   expression;
parser\_t       parser;

double x0 = 123.0;
double x1 = 678.0;

std::string expression\_string = "x + 1";

symbol\_table0.add\_variable("x",x0);
symbol\_table1.add\_variable("x",x1);

expression.register\_symbol\_table(symbol\_table0);
expression.register\_symbol\_table(symbol\_table1);

parser.compile(expression\_string,expression);

expression.value(); // 123 + 1


The symbol table supports  adding references to external  instances of
types  that  can  be accessed  within  expressions  via the  following
methods:

1. bool add\_variable (const std::string\& name,       scalar\_t\&)
2. bool add\_constant (const std::string\& name, const scalar\_t\&)
3. bool add\_stringvar(const std::string\& name,    std::string\&)
4. bool add\_vector   (const std::string\& name,    vector\_type\&)


Note: The 'vector' type must  be comprised from a contiguous  array of
scalars with a size that is  larger than zero. The vector type  itself
can be any one of the following:

1. std::vector<scalar\_t>
2. scalar\_t(\&v)[N]
3. scalar\_t* and array size
4. exprtk::vector\_view<scalar\_t>


When  registering  a variable,  vector,  string or  function  with  an
instance of a symbol\_table, the call to 'add\_...' may fail and  return
a false result due to one or more of the following reasons:

1. Variable name contains invalid characters or is ill-formed
2. Variable name conflicts with a reserved word (eg: 'while')
3. Variable name conflicts with a previously registered variable
4. A vector of size (length) zero is being registered
5. A free function exceeding fifteen parameters is being registered
6. The symbol\_table instance is in an invalid state


(2) Expression
A structure that holds an abstract syntax tree or AST for a  specified
expression and is used to evaluate said expression. Evaluation of  the
expression is accomplished by performing a post-order traversal of the
AST.  If  a  compiled  Expression  uses  variables  or  user   defined
functions, it will have an associated Symbol Table, which will contain
references  to said  variables, functions  or strings. An example  AST
structure for the denoted expression is as follows:

Expression:  z := (x + y\^-2.345) * sin(pi / min(w - 7.3,v))

[Root]
|
[Assignment]

The above denoted AST will be evaluated in the following order:

(01) Load Variable  (z)        (10) Load Constant  (7.3)
(02) Load Variable  (x)        (11) Subtraction    (09 \& 10)
(03) Load Variable  (y)        (12) Load Variable  (v)
(04) Load Constant  (2.345)    (13) Min            (11 \& 12)
(05) Negation       (04)       (14) Division       (08 \& 13)
(06) Exponentiation (03 \& 05)  (15) Sin            (14)
(07) Addition       (02 \& 06)  (16) Multiplication (07 \& 15)
(08) Load Constant  (pi)       (17) Assignment     (01 \& 16)
(09) Load Variable  (w)


Generally an expression in ExprTk can be thought of as a free function
similar to those  found in imperative  languages. This form  of pseudo
function will have a name, it may have a set of one or more inputs and
will return at least one value as its result. Futhermore the  function
when invoked, may  cause a side-effect  that changes the  state of the
host program.

As an  example the  following is  a pseudo-code  definition of  a free
function that performs a computation taking four inputs, modifying one
of them and returning a value based on some arbitrary calculation:

ResultType foo(InputType x, InputType y, InputType z, InputType w)
{
w = 2 * x\^y + z;        // Side-Effect
return abs(x - y) / z;  // Return Result
}


Given the above definition the following is a functionally  equivalent
version using ExprTk:

const std::string foo\_str = " w := 2 * x\^y + z; "
" abs(x - y) / z;   ";

T x, y, z, w;

symbol\_table\_t symbol\_table;
symbol\_table.add\_variable("x",x);
symbol\_table.add\_variable("y",y);
symbol\_table.add\_variable("z",z);
symbol\_table.add\_variable("w",w);

expression\_t foo;
foo.register\_symbol\_table(symbol\_table);

parser\_t parser;
if (!parser.compile(foo\_str,foo))
{
// Error in expression...
return;
}

T result = foo.value();


(3) Parser
A  component  which  takes  as input  a  string  representation  of an
expression and attempts to compile said input with the result being an
instance  of  Expression.  If  an  error  is  encountered  during  the
compilation  process, the  parser will  stop compiling  and return  an
error status code,  with a more  detailed description of  the error(s)
and  its  location  within  the  input  provided  by  the  'get\_error'
interface.


Note: The exprtk::expression  and exprtk::symbol\_table components  are
reference counted entities. Copy constructing or assigning to or  from
either component will result in  a shallow copy and a  reference count
increment,  rather  than  a  complete  replication.  Furthermore   the
expression  and symbol\_table  components being  Default-Constructible,
Copy-Constructible  and  Copy-Assignable  make  them  compatible  with
various  C++  standard  library   containers  and  adaptors  such   as
std::vector, std::map, std::stack etc.

The following is  an example of  two unique expressions,  after having
being instantiated and  compiled, one expression  is  assigned to  the
other. The diagrams depict  their initial and post  assignment states,
including  which  control  block each  expression references and their
associated reference counts.


exprtk::expression e0; // constructed expression, eg: x + 1
exprtk::expression e1; // constructed expression, eg: 2z + y

+-----[ e0 cntrl block]----+     +-----[ e1 cntrl block]-----+
| 1. Expression Node 'x+1' |     | 1. Expression Node '2z+y' |
| 2. Ref Count: 1          |<-+  | 2. Ref Count: 1           |<-+
+--------------------------+  |  +---------------------------+  |
|                                 |
+--[ e0 expression]--+      |    +--[ e1 expression]--+       |
| 1. Reference to    ]------+    | 1. Reference to    ]-------+
| e0 Control Block   |           | e1 Control Block   |
+--------------------+           +--------------------+


e0 = e1; // e0 and e1 are now 2z+y

+-----[ e1 cntrl block]-----+
| 1. Expression Node '2z+y' |
+----------->| 2. Ref Count: 2           |<----------+
|            +---------------------------+           |
|                                                    |
|   +--[ e0 expression]--+  +--[ e1 expression]--+   |
+---[ 1. Reference to    |  | 1. Reference to    ]---+
| e1 Control Block   |  | e1 Control Block   |
+--------------------+  +--------------------+

The reason for  the above complexity  and restrictions of  deep copies
for the expression and symbol\_table components is because  expressions
may include user defined variables or functions. These are embedded as
references into the expression's AST. When copying an expression, said
references  need to  also  be  copied. If  the references  are blindly
copied,  it  will then  result  in two  or more  identical expressions
utilizing the exact same  references for variables. This  obviously is
not the  default assumed  scenario and  will give  rise to non-obvious
behaviours  when  using the  expressions in  various contexts such  as
muli-threading et al.

The prescribed method for cloning an expression is to compile it  from
its string  form. Doing so will allow the 'user' to  properly consider
the exact source of user defined variables and functions.

Note:  The  exprtk::parser  is  a  non-copyable  and  non-thread  safe
component, and should only be shared via either a reference, a  shared
pointer  or  a  std::ref  mechanism,  and  considerations  relating to
synchronisation  taken  into  account  where  appropriate.  The parser
represents an object factory,  specifically a factory of  expressions,
and generally should  not be instantiated  solely on a  per expression
compilation basis.

The  following  diagram  and  example depicts  the  flow  of  data and
operations  for  compiling  multiple expressions  via  the  parser and
inserting  the  newly  minted  exprtk::expression  instances  into   a
std::vector.

+----[exprtk::parser]---+
|   Expression Factory  |
| parser\_t::compile(...)|
+--> ~.~.~.~.~.~.~.~.~.~ ->--+
| +-----------------------+  |
Expressions in     |                            |  Expressions as
string form        A                            V  exprtk::expression
|                            |  instances
[s0:'x+1']--->--+  |                            |  +-[e0: x+1]
|  |                            |  |
[s1:'2z+y']-->--+--+                            +->+-[e1: 2z+y]
|                                  |
[s2:'sin(k+w)']-+                                  +-[e2: sin(k+w)]


const std::string expression\_str[3]
= \{ "x + 1", "2x + y", "sin(k + w)" \};

std::vector<expression\_t> expression\_list;

parser\_t             parser;
expression\_t     expression;
symbol\_table\_t symbol\_table;

expression.register\_symbol\_table(symbol\_table);

for (std::size\_t i = 0; i < 3; ++i)
{
if (parser.compile(expression\_str[i],expression))
{
	expression\_list.push\_back(expression);
}
else
std::cout << "Error in " << expression\_str[i] << "\\n";
}

for (auto\& e : expression\_list)
{
e.value();
}

~~~~~~~~~~~~~~~~~~~~~~~~~~~~~~~~~~~~~~~~~~~~~~~~~~~~~~~~~~

[SECTION 11 - COMPILATION OPTIONS]
The exprtk::parser  when being  instantiated takes  as input  a set of
options  to be  used during  the compilation  process of  expressions.
An  example instantiation  of exprtk::parser  where only  the  joiner,
commutative and strength reduction options are enabled is as  follows:

typedef exprtk::parser<NumericType>::settings\_t settings\_t;

std::size\_t compile\_options = settings\_t::e\_joiner            +
settings\_t::e\_commutative\_check +
settings\_t::e\_strength\_reduction;

parser\_t parser(compile\_options);


Currently  seven  types of  compile  time options  are  supported, and
enabled by default. The options and their explanations are as follows:

(1) Replacer
(2) Joiner
(3) Numeric Check
(4) Bracket Check
(5) Sequence Check
(6) Commutative Check
(7) Strength Reduction Check


(1) Replacer (e\_replacer)
Enable replacement of specific  tokens with other tokens.  For example
the token  "true" of  type symbol  will be  replaced with  the numeric
token of value one.

(a) (x < y) == true   --->  (x < y) == 1
(b) false == (x > y)  --->  0 == (x > y)


(2) Joiner (e\_joiner)
Enable  joining  of  multi-character  operators  that  may  have  been
incorrectly  disjoint in the  string  representation  of the specified
expression. For example the consecutive tokens of ">" "=" will  become
">=" representing  the "greater  than or  equal to"  operator. If  not
properly resolved the  original form will  cause a compilation  error.
The  following  is  a listing  of the  scenarios that  the joiner  can
handle:

(a) '>' '='  --->  '>='  (gte)
(b) '<' '='  --->  '<='  (lte)
(c) '=' '='  --->  '=='  (equal)
(d) '!' '='  --->  '!='  (not-equal)
(e) '<' '>'  --->  '<>'  (not-equal)
(f) ':' '='  --->  ':='  (assignment)
(g) '+' '='  --->  '+='  (addition assignment)
(h) '-' '='  --->  '-='  (subtraction assignment)
(i) '*' '='  --->  '*='  (multiplication assignment)
(j) '/' '='  --->  '/='  (division assignment)
(k) '\%' '='  --->  '\%='  (modulo assignment)
(l) '+' '-'  --->  '-'   (subtraction)
(m) '-' '+'  --->  '-'   (subtraction)
(n) '-' '-'  --->  '+'   (addition)
(o) '<=' '>' --->  '<=>' (swap)


An example of the transformation that takes place is as follows:

(a) (x > = y) and (z ! = w)  --->  (x >= y) and (z != w)


(3) Numeric Check (e\_numeric\_check)
Enable validation of tokens representing numeric types so as to  catch
any errors prior  to the costly  process of the  main compilation step
commencing.


(4) Bracket Check (e\_bracket\_check)
Enable  the  check for  validating  the ordering  of  brackets in  the
specified expression.


(5) Sequence Check (e\_sequence\_check)
Enable the  check for  validating that  sequences of  either pairs  or
triplets of tokens make sense.  For example the following sequence  of
tokens when encountered will raise an error:

(a) (x + * 3)  --->  sequence error


(6) Commutative Check (e\_commutative\_check)
Enable the check that will transform sequences of pairs of tokens that
imply a multiplication operation.  The following are some  examples of
such transformations:

(a) 2x             --->  2 * x
(b) 25x\^3          --->  25 * x\^3
(c) 3(x + 1)       --->  3 * (x + 1)
(d) (x + 1)4       --->  (x + 1) * 4
(e) 5foo(x,y)      --->  5 * foo(x,y)
(f) foo(x,y)6 + 1  --->  foo(x,y) * 6 + 1
(g) (4((2x)3))     --->  4 * ((2 * x) * 3)
(h) w(x) + (y)z    --->  w * x + y * z


(7) Strength Reduction Check (e\_strength\_reduction)
Enable  the  use  of  strength  reduction  optimisations  during   the
compilation  process.  In  ExprTk  strength  reduction   optimisations
predominantly involve  transforming sub-expressions  into other  forms
that  are algebraically  equivalent yet  less costly  to compute.  The
following are examples of the various transformations that can occur:

(a) (x / y) / z        --->  x / (y * z)
(b) (x / y) / (z / w)  --->  (x * w) / (y * z)
(c) (2 * x) - (2 * y)  --->  2 * (x - y)
(d) (2 / x) / (3 / y)  --->  (2 / 3) / (x * y)
(e) (2 * x) * (3 * y)  --->  (2 * 3) * (x * y)


Note:
When using  strength reduction  in conjunction  with expressions whose
inputs or sub-expressions may result  in values nearing either of  the
bounds of the underlying numeric  type (eg: double), there may  be the
possibility of a decrease in the precision of results.

In  the following  example the  given expression  which represents  an
attempt at computing the average  between x and y will  be transformed
as follows:

(0.5 * x) + (y * 0.5) ---> 0.5 * (x + y)

There  may be  situations where  the above  transformation will  cause
numerical overflows and  that the original  form of the  expression is
desired over the strength reduced form. In these situations it is best
to turn off strength reduction optimisations  or to use a type with  a
larger numerical bound.

~~~~~~~~~~~~~~~~~~~~~~~~~~~~~~~~~~~~~~~~~~~~~~~~~~~~~~~~~~

[SECTION 12 - EXPRESSION STRUCTURES]
Exprtk supports mathematical expressions in numerous forms based on  a
simple  imperative  programming  model. This  section  will  cover the
following  topics  related  to general  structure  and  programming of
expression using ExprTk:

(1) Multi-Statement Expressions
(2) Statements And Side-Effects
(3) Conditional Statements
(4) Special Functions


(1) Multi-Statement Expressions
Expressions in ExprTk can be  comprised of one more statements,  which
may  sometimes  be  called  sub-expressions.  The  following  are  two
examples of expressions stored  in std::string variables, the  first a
single statement and the second a multi-statement expression:

std::string single\_statement = " z := x + y ";

std::string multi\_statement  = " var temp := x; "
" x := y + z;    "
" y := temp;     ";


In a  multi-statement expression,  the final  statement will determine
the overall result of the expression. In the following multi-statement
expression, the result of the expression when evaluated will be '2.3',
which will also be the value stored in the 'y' variable.

z := x + y;
y := 2.3;


As  demonstrated  in  the  expression  above,  statements  within   an
expression are  separated using  the semi-colon  ';' operator.  In the
event  two  statements are  not  separated by  a  semi-colon, and  the
implied multiplication  feature is  active (enabled  by default),  the
compiler  will  assume  a  multiplication  operation  between  the two
statements.

In the following example we have a multi-statement expression composed
of two variable definitions and initialisations for variables x and  y
and two seemingly separate mathematical operations.

var x:= 2;
var y:= 3;
x + 1
y * 2


However the result of  the expression will not  be 6 as may  have been
assumed based on  the calculation of  'y * 2',  but rather the  result
will be 8. This  is because the compiler  will have conjoined the  two
mathematical statements into one  via a multiplication operation.  The
expression when compiled will actually evaluate as the following:

var x:= 2;
var y:= 3;
x + 1 * y * 2;   // 2 + 1 * 3 * 2 == 8


In ExprTk any valid statement  will itself return a value.  This value
can  further  be  used  in  conjunction  with  other  statements. This
includes language structures such as if-statements, loops (for, while)
and the switch statement. Typically the last statement executed in the
given construct  (conditional, loop  etc), will  be the  value that is
returned.

In the following example, the  return value of the expression  will be
11, which is the sum of the variable 'x' and the final value  computed
within the loop body on its last iteration:

var x := 1;
x + for (var i := x; i < 10; i += 1)
{
i / 2;
i + 1;
}


(2) Statements And Side-Effects
Statements themselves may have side effects, which in-turn effect  the
proceeding statements in multi-statement expressions.

A statement is said  to have a side-effect  if it causes the  state of
the  expression to  change in  some way  -  this  includes but  is not
limited to the  modification of the  state of external  variables used
within the expression. Currently  the following actions being  present
in a statement will cause it to have a side-effect:

(a) Assignment operation (explicit or potentially)
(b) Invoking a user-defined function that has side-effects

The following are examples of expressions where the side-effect status
of the statements (or sub-exressions) within the expressions have been
noted:

+-+----------------------+------------------------------+
|\#|      Expression      |      Side Effect Status      |
+-+----------------------+------------------------------+
|0| x + y                | False                        |
+-+----------------------+------------------------------+
|1| z := x + y           | True - Due to assignment     |
+-+----------------------+------------------------------+
|2| abs(x - y)           | False                        |
+-+----------------------+------------------------------+
|3| abs(x - y);          | False                        |
| | z := (x += y);       | True - Due to assignments    |
+-+----------------------+------------------------------+
|4| abs(x - y);          | False                        |
| | z := (x += y);       | True - Due to assignments    |
+-+----------------------+------------------------------+
|5| var t := abs(x - y); | True - Due to initialisation |
| | t + x;               | False                        |
| | z := (x += y);       | True - Due to assignments    |
+-+----------------------+------------------------------+
|6| foo(x - y)           | True - user defined function |
+-+----------------------+------------------------------+


Note: In example 6 from the above set, it is assumed the user  defined
function foo has been registered  as having a side\_effect. By  default
all user defined  functions are assumed  to have side-effects,  unless
they are  configured in  their constructors  to not  have side-effects
using   the   'disable\_has\_side\_effects'  free   function.   For  more
information review  Section 15 - User Defined Functions  sub-section 7
Function Side-Effects.

At this point we  can see that there  will be expressions composed  of
certain kinds  of statements  that when  executed will  not effect the
nature  of the  expression's  result.  These statements  are typically
called 'dead code'.  These statements though  not effecting the  final
result will still be executed and as such they will consume processing
time that could otherwise be saved. As such ExprTk attempts to  detect
and remove such statements from expressions.

The  'Dead  Code  Elimination' (DCE)  optimisation  process,  which is
enabled by default, will remove any statements that are determined  to
not have a side effect in a multi-statement expression, excluding  the
final or last statement.

By default the final statement in an expression will always be present
regardless of  its side-effect  status, as  it is  the statement whose
value will be used as the result of the expression.

In order to further explain the actions taken during the DCE  process,
lets review the following expression:

var x := 2;      // Statement 1
var y := x + 2;  // Statement 2
x + y;           // Statement 3
y := x + 3y;     // Statement 4
x - y;           // Statement 5


The above expression has five statements.  Three of them (1, 2 and  4)
actively have side-effects. The first two are variable declaration and
initialisations, where as the third is due to an assignment operation.
There  are  two  statements (3  and 5),  that do  not explicitly  have
side-effects, however the latter, statement 5, is the final  statement
in the expression and hence will be assumed to have a side-effect.

During compilation when the DCE  optimisation is applied to the  above
expression, statement 3 will be removed from the expression, as it has
no  bearing  on  the  final result  of  expression,  the  rest of  the
statements will all remain. The optimised form of the expression is as
follows:

var x := 2;      // Statement 1
var y := x + 2;  // Statement 2
y := x + 3y;     // Statement 3
x - y;           // Statement 4


(3) Conditional Statements (If-Then-Else)
ExprTk support two forms  of conditional branching or  otherwise known
as  if-statements.  The  first  form,  is  a  simple  function   based
conditional  statement, that  takes exactly  three input  expressions:
condition, consequent  and alternative.  The following  is an  example
expression that utilises the function based if-statement.

x := if (y < z, y + 1, 2 * z)


In the  example above,  if the  condition 'y  < z'  is true,  then the
consequent 'y + 1' will be evaluated, its value will be  returned  and
subsequently assigned to the  variable 'x'. Otherwise the  alternative
'2 *  z' will  be evaluated  and its  value will  be returned. This is
essentially the  simplest form of  an if-then-else statement. A simple
variation of  the expression  where the  value of  the if-statement is
used within another statement is as follows:

x := 3 * if (y < z, y + 1, 2 * z) / 2


The second form of if-statement resembles the standard syntax found in
most imperative languages. There are two variations of the statement:

(a) If-Statement
(b) If-Then-Else Statement


(a) If-Statement
This version  of the  conditional statement  returns the  value of the
consequent expression when the  condition expression is true,  else it
will return a quiet NaN value as its result.

Example 1:
x := if (y < z) y + 3;

Example 2:
x := if (y < z)
{
y + 3
}

The two  example expressions  above are  equivalent. If  the condition
'y < z' is true, the  'x' variable will be  assigned the value of  the
consequent 'y + 3', otherwise it  will be assigned the value of  quiet
NaN.  As  previously  discussed,  if-statements  are  value  returning
constructs, and if  not properly terminated  using a semi-colon,  will
end-up  combining  with  the  next  statement  via  a   multiplication
operation. The following example will NOT result in the expected value
of 'w + x' being returned:

x := if (y < z) y + 3  // missing semi-colon ';'
w + x


When the  above supposed  multi-statement expression  is compiled, the
expression  will  have  a  multiplication  inserted  between  the  two
'intended' statements resulting in the unanticipated expression:

x := (if (y < z) y + 3) * w + x


The  solution  to  the  above situation  is  to  simply  terminate the
conditional statement with a semi-colon as follows:

x := if (y < z) y + 3;
w + x


(b) If-Then-Else Statement
The second variation of  the if-statement is to  allow for the use  of
Else and If-Else cascading statements. Examples of such statements are
as follows:

Example 1:             Example 2:         Example 3:
if (x < y)             if (x < y)         if (x > y + 1)
z := x + 3;          {                    y := abs(x - z);
else                     y := z + x;      else
y := x - z;            z := x + 3;      {
}                    y := z + x;
else                 z := x + 3;
y := x - z;      };


Example 4:             Example 5:         Example 6:
if (2 * x < max(y,3))  if (x < y)         if (x < y or (x + z) > y)
{                        z := x + 3;      {
	y := z + x;          else if (2y != z)    z := x + 3;
	z := x + 3;          {                    y := x - z;
	}                        z := x + 3;      }
else if (2y - z)         y := x - z;      else if (abs(2y - z) >= 3)
y := x - z;          }                    y := x - z;
else               else
x * x;           {
z := abs(x * x);
x * y * z;
};


In  the case  where  there  is no  final else  statement and  the flow
through the conditional  arrives at this  final point, the  same rules
apply to this form of if-statement as to the previous. That is a quiet
NaN will be  returned as the  result of the  if-statement. Furthermore
the same requirements of  terminating the statement with  a semi-colon
apply.

(4) Special Functions
The purpose  of special  functions in  ExprTk is  to provide  compiler
generated equivalents of common mathematical expressions which can  be
invoked by  using the  'special function'  syntax (eg:  \$f12(x,y,z) or
\$f82(x,y,z,w)).

Special functions dramatically decrease  the total evaluation time  of
expressions which would otherwise  have been written using  the common
form by reducing the total number  of nodes in the evaluation tree  of
an  expression  and  by  also  leveraging  the  compiler's  ability to
correctly optimise such expressions for a given architecture.

3-Parameter                       4-Parameter
+-------------+-------------+    +--------------+------------------+
|  Prototype  |  Operation  |    |  Prototype   |    Operation     |
+-------------+-------------+    +--------------+------------------+
\$f00(x,y,z) | (x + y) / z       \$f48(x,y,z,w) | x + ((y + z) / w)
\$f01(x,y,z) | (x + y) * z       \$f49(x,y,z,w) | x + ((y + z) * w)
\$f02(x,y,z) | (x + y) - z       \$f50(x,y,z,w) | x + ((y - z) / w)
\$f03(x,y,z) | (x + y) + z       \$f51(x,y,z,w) | x + ((y - z) * w)
\$f04(x,y,z) | (x - y) + z       \$f52(x,y,z,w) | x + ((y * z) / w)
\$f05(x,y,z) | (x - y) / z       \$f53(x,y,z,w) | x + ((y * z) * w)
\$f06(x,y,z) | (x - y) * z       \$f54(x,y,z,w) | x + ((y / z) + w)
\$f07(x,y,z) | (x * y) + z       \$f55(x,y,z,w) | x + ((y / z) / w)
\$f08(x,y,z) | (x * y) - z       \$f56(x,y,z,w) | x + ((y / z) * w)
\$f09(x,y,z) | (x * y) / z       \$f57(x,y,z,w) | x - ((y + z) / w)
\$f10(x,y,z) | (x * y) * z       \$f58(x,y,z,w) | x - ((y + z) * w)
\$f11(x,y,z) | (x / y) + z       \$f59(x,y,z,w) | x - ((y - z) / w)
\$f12(x,y,z) | (x / y) - z       \$f60(x,y,z,w) | x - ((y - z) * w)
\$f13(x,y,z) | (x / y) / z       \$f61(x,y,z,w) | x - ((y * z) / w)
\$f14(x,y,z) | (x / y) * z       \$f62(x,y,z,w) | x - ((y * z) * w)
\$f15(x,y,z) | x / (y + z)       \$f63(x,y,z,w) | x - ((y / z) / w)
\$f16(x,y,z) | x / (y - z)       \$f64(x,y,z,w) | x - ((y / z) * w)
\$f17(x,y,z) | x / (y * z)       \$f65(x,y,z,w) | ((x + y) * z) - w
\$f18(x,y,z) | x / (y / z)       \$f66(x,y,z,w) | ((x - y) * z) - w
\$f19(x,y,z) | x * (y + z)       \$f67(x,y,z,w) | ((x * y) * z) - w
\$f20(x,y,z) | x * (y - z)       \$f68(x,y,z,w) | ((x / y) * z) - w
\$f21(x,y,z) | x * (y * z)       \$f69(x,y,z,w) | ((x + y) / z) - w
\$f22(x,y,z) | x * (y / z)       \$f70(x,y,z,w) | ((x - y) / z) - w
\$f23(x,y,z) | x - (y + z)       \$f71(x,y,z,w) | ((x * y) / z) - w
\$f24(x,y,z) | x - (y - z)       \$f72(x,y,z,w) | ((x / y) / z) - w
\$f25(x,y,z) | x - (y / z)       \$f73(x,y,z,w) | (x * y) + (z * w)
\$f26(x,y,z) | x - (y * z)       \$f74(x,y,z,w) | (x * y) - (z * w)
\$f27(x,y,z) | x + (y * z)       \$f75(x,y,z,w) | (x * y) + (z / w)
\$f28(x,y,z) | x + (y / z)       \$f76(x,y,z,w) | (x * y) - (z / w)
\$f29(x,y,z) | x + (y + z)       \$f77(x,y,z,w) | (x / y) + (z / w)
\$f30(x,y,z) | x + (y - z)       \$f78(x,y,z,w) | (x / y) - (z / w)
\$f31(x,y,z) | x * y\^2 + z       \$f79(x,y,z,w) | (x / y) - (z * w)
\$f32(x,y,z) | x * y\^3 + z       \$f80(x,y,z,w) | x / (y + (z * w))
\$f33(x,y,z) | x * y\^4 + z       \$f81(x,y,z,w) | x / (y - (z * w))
\$f34(x,y,z) | x * y\^5 + z       \$f82(x,y,z,w) | x * (y + (z * w))
\$f35(x,y,z) | x * y\^6 + z       \$f83(x,y,z,w) | x * (y - (z * w))
\$f36(x,y,z) | x * y\^7 + z       \$f84(x,y,z,w) | x*y\^2 + z*w\^2
\$f37(x,y,z) | x * y\^8 + z       \$f85(x,y,z,w) | x*y\^3 + z*w\^3
\$f38(x,y,z) | x * y\^9 + z       \$f86(x,y,z,w) | x*y\^4 + z*w\^4
\$f39(x,y,z) | x * log(y)+z      \$f87(x,y,z,w) | x*y\^5 + z*w\^5
\$f40(x,y,z) | x * log(y)-z      \$f88(x,y,z,w) | x*y\^6 + z*w\^6
\$f41(x,y,z) | x * log10(y)+z    \$f89(x,y,z,w) | x*y\^7 + z*w\^7
\$f42(x,y,z) | x * log10(y)-z    \$f90(x,y,z,w) | x*y\^8 + z*w\^8
\$f43(x,y,z) | x * sin(y)+z      \$f91(x,y,z,w) | x*y\^9 + z*w\^9
\$f44(x,y,z) | x * sin(y)-z      \$f92(x,y,z,w) | (x and y) ? z : w
\$f45(x,y,z) | x * cos(y)+z      \$f93(x,y,z,w) | (x or  y) ? z : w
\$f46(x,y,z) | x * cos(y)-z      \$f94(x,y,z,w) | (x <   y) ? z : w
\$f47(x,y,z) | x ? y : z         \$f95(x,y,z,w) | (x <=  y) ? z : w
\$f96(x,y,z,w) | (x >   y) ? z : w
\$f97(x,y,z,w) | (x >=  y) ? z : w
\$f98(x,y,z,w) | (x ==  y) ? z : w
\$f99(x,y,z,w) | x*sin(y)+z*cos(w)

~~~~~~~~~~~~~~~~~~~~~~~~~~~~~~~~~~~~~~~~~~~~~~~~~~~~~~~~~~

[SECTION 13 - VARIABLE, VECTOR \& STRING DEFINITION]
ExprTk supports the definition of expression local variables,  vectors
and  strings.  The definitions  must  be unique  as  shadowing is  not
allowed and object life-times are based on scope. Definitions use  the
following general form:

var <name> := <initialiser>;

(1) Variable Definition
Variables are  of numeric  type denoting  a single  value. They can be
explicitly initialised to a value, otherwise they will be defaulted to
zero. The following are examples of variable definitions:

(a) Initialise x to zero
var x;

(b) Initialise y to three
var y := 3;

(c) Initialise z to the expression
var z := if (max(1, x + y) > 2, w, v);


(2) Vector Definition
Vectors are arrays of a common numeric type. The elements in a  vector
can be explicitly initialised, otherwise they will all be defaulted to
zero. The following are examples of vector definitions:

(a) Initialise all values to zero
var x[3];

(b) Initialise all values to zero
var x[3] := {};

(c) Initialise all values to given expression
var x[3] := [123 + 3y + sin(w / z)];

(d) Initialise the first two values, all other elements to zero
var x[3] := { 1 + x[2], sin(y[0] / x[]) + 3 };

(e) Initialise the first three (all) values
var x[3] := { 1, 2, 3 };

(f) Initialise vector from a vector
var x[4] := { 1, 2, 3, 4 };
var y[3] := x;

(g) Initialise vector from a smaller vector
var x[3] := { 1, 2, 3 };
var y[5] := x;   // 1, 2, 3, ??, ??

(h) Non-initialised vector
var x[3] := null; // ?? ?? ??

(i) Error as there are too many initialisers
var x[3] := { 1, 2, 3, 4 };

(j) Error as a vector of size zero is not allowed.
var x[0];


(3) String Definition
Strings are sequences comprised of 8-bit characters. They can only be
defined  with an explicit  initialisation  value. The  following  are
examples of string variable definitions:

(a) Initialise to a string
var x := 'abc';

(b) Initialise to an empty string
var x := '';

(c) Initialise to a string expression
var x := 'abc' + '123';

(d) Initialise to a string range
var x := 'abc123'[2:4];

(e) Initialise to another string variable
var x := 'abc';
var y := x;

(f) Initialise to another string variable range
var x := 'abc123';
var y := x[2:4];

(g) Initialise to a string expression
var x := 'abc';
var y := x + '123';

(h) Initialise to a string expression range
var x := 'abc';
var y := (x + '123')[1:3];


(4) Return Value
Variable and vector  definitions have a  return value. In  the case of
variable definitions, the value  to which the variable  is initialised
will be returned. Where as for vectors, the value of the first element
(eg: v[0]) will be returned.

8 == ((var x := 7;) + 1)
4 == (var y[3] := {4, 5, 6};)


(5) Variable/Vector Assignment
The value of a variable can be assigned to a vector and a vector or  a
vector expression can be assigned to a variable.

(a) Variable To Vector:
Every element of the vector is assigned the value of the variable
or expression.
var x    := 3;
var y[3] := { 1, 2, 3 };
y := x + 1;

(b) Vector To Variable:
The variable is assigned the value of the first element of the
vector (aka vec[0])
var x    := 3;
var y[3] := { 1, 2, 3 };
x := y + 1;


Note: During the expression  compilation phase, tokens are  classified
based on the following priorities:

(a) Reserved keywords or operators (+, -, and, or, etc)
(b) Base functions (abs, sin, cos, min, max etc)
(c) Symbol table variables
(d) Expression local defined variables
(e) Symbol table functions
(f) Unknown symbol resolver based variables

~~~~~~~~~~~~~~~~~~~~~~~~~~~~~~~~~~~~~~~~~~~~~~~~~~~~~~~~~~

[SECTION 14 - VECTOR PROCESSING]
ExprTk  provides  support  for   various  forms  of  vector   oriented
arithmetic, inequalities and  processing. The various  supported pairs
are as follows:

(a) vector and vector (eg: v0 + v1)
(b) vector and scalar (eg: v  + 33)
(c) scalar and vector (eg: 22 *  v)

The following is a list of operations that can be used in  conjunction
with vectors:

(a) Arithmetic:       +, -, *, /, \%
(b) Exponentiation:   vector \^ scalar
(c) Assignment:       :=, +=, -=, *=, /=, \%=, <=>
(d) Inequalities:     <, <=, >, >=, ==, =, equal
(e) Boolean logic:    and, nand, nor, or, xnor, xor
(f) Unary operations:
abs, acos, acosh, asin, asinh, atan, atanh, ceil, cos,  cosh,
cot, csc,  deg2grad, deg2rad,  erf, erfc,  exp, expm1, floor,
frac, grad2deg, log, log10, log1p, log2, rad2deg, round, sec,
sgn, sin, sinc, sinh, sqrt, swap, tan, tanh, trunc
(g) Aggregate and Reduce operations:
avg, max, min, mul,  dot, dotk, sum, sumk,  count, all\_true,
all\_false, any\_true, any\_false
(h) Transformation operations:
copy, rotate-left/right, shift-left/right, sort, nth\_element
(i) BLAS-L1:
axpy, axpby, axpyz, axpbyz, axpbz

Note: When one of  the above  described operations  is being performed
between two  vectors, the  operation will  only span  the size  of the
smallest vector.  The elements  of the  larger vector  outside of  the
range will  not be included. The  operation  itself will  be processed
element-wise over values the smaller of the two ranges.

The  following  simple  example  demonstrates  the  vector  processing
capabilities by computing the dot-product of the vectors v0 and v1 and
then assigning it to the variable v0dotv1:

var v0[3] := { 1, 2, 3 };
var v1[3] := { 4, 5, 6 };
var v0dotv1 := sum(v0 * v1);


The following is a for-loop based implementation that is equivalent to
the previously mentioned dot-product computation expression:

var v0[3] := { 1, 2, 3 };
var v1[3] := { 4, 5, 6 };
var v0dotv1;

for (var i := 0; i < min(v0[],v1[]); i += 1)
{
v0dotv1 += (v0[i] * v1[i]);
}


Note: When  the aggregate or reduction  operations denoted  above  are
used  in conjunction with a  vector or  vector  expression, the return
value is not a vector but rather a single value.

var x[3] := { 1, 2, 3 };

sum(x)      ==  6
sum(1 + 2x) == 15
avg(3x + 1) ==  7
min(1 / x)  == (1 / 3)
max(x / 2)  == (3 / 2)
sum(x > 0 and x < 5) == x[]


When utilizing external user defined  vectors via the symbol table  as
opposed to expression local defined vectors, the typical  'add\_vector'
method from the symbol table will register the entirety of the  vector
that is passed. The following example attempts to evaluate the sum  of
elements of  the external  user defined  vector within  a typical  yet
trivial expression:

std::string reduce\_program = " sum(2 * v + 1) ";

std::vector<T> v0 { T(1.1), T(2.2), ..... , T(99.99) };

symbol\_table\_t symbol\_table;
symbol\_table.add\_vector("v",v);

expression\_t expression;
expression.register\_symbol\_table(symbol\_table);

parser\_t parser;
parser.compile(reduce\_program,expression);

T sum = expression.value();


For the most part, this is  a very common use-case. However there  may
be situations where one may want to evaluate the same vector  oriented
expression many times over, but using different vectors or sub  ranges
of the same vector of the same size to that of the original upon every
evaluation.

The usual solution is to  either recompile the expression for  the new
vector instance, or to  copy the contents from  the new vector to  the
symbol table registered vector  and then perform the  evaluation. When
the  vectors are  large or  the re-evaluation  attempts are  numerous,
these  solutions  can  become  rather  time  consuming  and  generally
inefficient.

std::vector<T> v1 { T(2.2), T(2.2), ..... , T(2.2) };
std::vector<T> v2 { T(3.3), T(3.3), ..... , T(3.3) };
std::vector<T> v3 { T(4.4), T(4.4), ..... , T(4.4) };

std::vector<std::vector<T>> vv { v1, v2, v3 };
...
T sum = T(0);

for (auto\& new\_vec : vv)
{
v = new\_vec; // update vector
sum += expression.value();
}


A  solution  to  the  above  'efficiency'  problem,  is  to  use   the
exprtk::vector\_view  object. The  vector\_view is  instantiated with  a
size and backing based upon a vector. Upon evaluations if the  backing
needs  to  be  'updated' to  either another  vector or  sub-range, the
vector\_view instance  can be  efficiently rebased,  and the expression
evaluated as normal.

exprtk::vector\_view<T> view = exprtk::make\_vector\_view(v,v.size());

symbol\_table\_t symbol\_table;
symbol\_table.add\_vector("v",view);

...

T sum = T(0);

for (auto\& new\_vec : vv)
{
view.rebase(new\_vec.data()); // update vector
sum += expression.value();
}

~~~~~~~~~~~~~~~~~~~~~~~~~~~~~~~~~~~~~~~~~~~~~~~~~~~~~~~~~~

[SECTION 15 - USER DEFINED FUNCTIONS]
ExprTk provides a means  whereby custom functions can  be defined  and
utilised within  expressions.  The   concept  requires  the  user   to
provide a reference  to the function  coupled with an  associated name
that  will be invoked within  expressions. Functions may take numerous
inputs but will always return a single value of the underlying numeric
type.

During  expression  compilation  when required  the  reference  to the
function  will be  obtained from  the associated  symbol\_table and  be
embedded into the expression.

There are five types of function interface:

+---+----------------------+-------------+----------------------+
| \# |         Name         | Return Type | Input Types          |
+---+----------------------+-------------+----------------------+
| 1 | ifunction            | Scalar      | Scalar               |
| 2 | ivararg\_function     | Scalar      | Scalar               |
| 3 | igeneric\_function    | Scalar      | Scalar,Vector,String |
| 4 | igeneric\_function II | String      | Scalar,Vector,String |
| 5 | function\_compositor  | Scalar      | Scalar               |
+---+----------------------+-------------+----------------------+

(1) ifunction
This interface supports zero to 20 input parameters of only the scalar
type (numbers). The  usage requires a custom function be  derived from
ifunction and to override one of the 21 function operators. As part of
the constructor the custom function will define how many parameters it
expects  to  handle.  The  following  example  defines  a  3 parameter
function called 'foo':

template <typename T>
struct foo : public exprtk::ifunction<T>
{
foo() : exprtk::ifunction<T>(3)
{}

T operator()(const T\& v1, const T\& v2, const T\& v3)
{
	return T(1) + (v1 * v2) / T(v3);
}
};


(2) ivararg\_function
This interface supports a variable number of scalar arguments as input
into the function. The function operator interface uses a  std::vector
specialised upon type T to facilitate parameter passing. The following
example defines a vararg function called 'boo':

template <typename T>
struct boo : public exprtk::ivararg\_function<T>
{
inline T operator()(const std::vector<T>\& arglist)
{
	T result = T(0);
	
	for (std::size\_t i = 0; i < arglist.size(); ++i)
	{
		result += arglist[i] / arglist[i > 0 ? (i - 1) : 0];
	}
	
	return result;
}
};


(3) igeneric\_function
This interface supports  a variable number  of arguments and  types as
input  into  the  function. The  function  operator  interface uses  a
std::vector  specialised  upon  the  type\_store  type  to   facilitate
parameter passing.

Scalar <-- function(i\_0, i\_1, i\_2....., i\_N)


The  fundamental  types  that  can  be  passed  into  the  function as
parameters and their views are as follows:

(1) Scalar - scalar\_view
(2) Vector - vector\_view
(3) String - string\_view


The above denoted type  views provide non-const reference-like  access
to each parameter, as such modifications made to the input  parameters
will  persist after  the function  call has  completed. The  following
example defines a generic function called 'too':

template <typename T>
struct too : public exprtk::igeneric\_function<T>
{
typedef typename exprtk::igeneric\_function<T>::parameter\_list\_t
parameter\_list\_t;

too()
{}

inline T operator()(parameter\_list\_t parameters)
{
	for (std::size\_t i = 0; i < parameters.size(); ++i)
	{
		...
	}
	
	return T(0);
}
};


In the example above, the input 'parameters' to the function operator,
parameter\_list\_t,  is  a  type  of  std::vector  of  type\_store.  Each
type\_store  instance  has  a  member  called  'type'  which  holds the
enumeration pertaining to the underlying type of the type\_store. There
are three type enumerations:

(1) e\_scalar - literals, variables, vector elements, expressions
eg: 123.456, x, vec[3x + 1], 2x + 3

(2) e\_vector - vectors, vector expressions
eg: vec1, 2 * vec1 + vec2 / 3

(3) e\_string - strings, string literals and range variants of both
eg: 'AString', s0, 'AString'[x:y], s1[1 + x:] + 'AString'


Each of the  parameters can be  accessed using its  designated view. A
typical loop for processing the parameters is as follows:

inline T operator()(parameter\_list\_t parameters)
{
typedef typename exprtk::igeneric\_function<T>::generic\_type
generic\_type;

typedef typename generic\_type::scalar\_view scalar\_t;
typedef typename generic\_type::vector\_view vector\_t;
typedef typename generic\_type::string\_view string\_t;

for (std::size\_t i = 0; i < parameters.size(); ++i)
{
	generic\_type\& gt = parameters[i];
	
	if (generic\_type::e\_scalar == gt.type)
	{
		scalar\_t x(gt);
		...
	}
	else if (generic\_type::e\_vector == gt.type)
	{
		vector\_t vector(gt);
		...
	}
	else if (generic\_type::e\_string == gt.type)
	{
		string\_t string(gt);
		...
	}
}

return T(0);
}


Most often than not a custom generic function will require a  specific
sequence of parameters, rather than some arbitrary sequence of  types.
In those situations, ExprTk can perform compile-time type checking  to
validate that function invocations  are carried out using  the correct
sequence of parameters. Furthermore  performing the checks at  compile
-time rather than at run-time (aka every time the function is invoked)
will result in expression evaluation performance gains.

Compile-time type  checking of  input parameters  can be  requested by
passing  a string  to the  constructor of  the igeneric\_function  that
represents the required sequence of parameter types. When no parameter
sequence is provided, it is implied the function can accept a variable
number of parameters comprised of any of the fundamental types.

Each fundamental type has an  associated character. The following is a
listing of said characters and their meanings:

(1) T - Scalar
(2) V - Vector
(3) S - String
(4) Z - Zero or no parameters
(5) ? - Any type (Scalar, Vector or String)
(6) * - Wildcard operator
(7) | - Parameter sequence delimiter


No other characters other than the seven denoted above may be included
in the parameter sequence  definition. If any such  invalid characters
do exist, registration of the associated generic function to a  symbol
table ('add\_function' method) will fail. If the parameter sequence  is
modified resulting in it becoming  invalid after having been added  to
the symbol table but before the compilation step, a compilation  error
will be incurred.

The  following   example  demonstrates   a  simple   generic  function
implementation with a user specified parameter sequence:

template <typename T>
struct moo : public exprtk::igeneric\_function<T>
{
typedef typename exprtk::igeneric\_function<T>::parameter\_list\_t
parameter\_list\_t;

moo()
: exprtk::igeneric\_function<T>("SVTT")
{}

inline T operator()(parameter\_list\_t parameters)
{
	...
}
};


In the example above the  generic function 'moo' expects exactly  four
parameters in the following sequence:

(1) String
(2) Vector
(3) Scalar
(4) Scalar

Note: The  'Z' or  no parameter option may not be  used in conjunction
with any other type option in a parameter sequence. When  incorporated
in the parameter sequence list, the no parameter option indicates that
the function may be invoked  without any parameters being passed.  For
more information refer to the section: 'Zero Parameter Functions'


(4) igeneric\_function II
This interface is identical to  the igeneric\_function, in that in  can
consume an  arbitrary number  of parameters  of varying  type, but the
difference being  that the  function returns  a string  and as such is
treated as a string when  invoked within expressions. As a  result the
function call can  alias a string  and interact with  other strings in
situations such as concatenation and equality operations.

String <-- function(i\_0, i\_1, i\_2....., i\_N)


The following example defines a generic function  named 'toupper' with
the string return type function operator being explicitly overridden:

template <typename T>
struct toupper : public exprtk::igeneric\_function<T>
{
typedef exprtk::igeneric\_function<T> igenfunct\_t
typedef typename igenfunct\_t::generic\_type generic\_t;
typedef typename igenfunct\_t::parameter\_list\_t parameter\_list\_t;
typedef typename generic\_t::string\_view string\_t;

toupper()
: exprtk::igeneric\_function<T>("S",igenfunct\_t::e\_rtrn\_string)
{}

inline T operator()(std::string\& result,
parameter\_list\_t parameters)
{
	result.clear();
	
	string\_t string(params[0]);
	
	for (std::size\_t i = 0; i < string.size(); ++i)
	{
		result += std::toupper(string[i]);
	}
	
	return T(0);
}
};


In the example above the  generic function 'toupper' expects only  one
input parameter  of type  string, as  noted by  the parameter sequence
string passed during the  constructor. Furthermore a second  parameter
is passed to the constructor indicating that it should be treated as a
string returning function -  by default it is  assumed to be a  scalar
returning function.

When executed,  the function  will return  as a  result a  copy of the
input string converted to uppercase form. An example expression  using
the toupper function registered as the symbol 'toupper' is as follows:

"'ABCDEF' == toupper('aBc') + toupper('DeF')"


Note: When adding a string type returning generic function to a symbol
table the  'add\_function' is invoked.  The example  below demonstrates
how this can be done:

toupper<T> tu;

exprtk::symbol\_table<T> symbol\_table;

symbol\_table.add\_function("toupper",tu);


Note: Two further  refinements to the  type checking facility  are the
possibilities  of  a variable  number  of common  types  which can  be
accomplished by using a wildcard '*' and a special 'any type' which is
done using  the '?'  character. It  should be  noted that the wildcard
operator is  associated with  the previous  type in  the sequence  and
implies one or more of that type.

template <typename T>
struct zoo : public exprtk::igeneric\_function<T>
{
typedef typename exprtk::igeneric\_function<T>::parameter\_list\_t
parameter\_list\_t;

zoo()
: exprtk::igeneric\_function<T>("SVT*V?")
{}

inline T operator()(parameter\_list\_t parameters)
{
	...
}
};


In the example above the generic function 'zoo' expects at least  five
parameters in the following sequence:

(1) String
(2) Vector
(3) One or more Scalars
(4) Vector
(5) Any type (one type of either a scalar, vector or string)


A final  piece of  type checking  functionality is  available for  the
scenarios where  a single  function name  is intended  to be  used for
multiple distinct parameter sequences,  another name for this  feature
is function  overloading. The  parameter sequences  are passed  to the
constructor as a  single string delimited  by the pipe  '|' character.
Two specific overrides of the  function operator are provided one  for
standard generic functions and one for string returning functions. The
overrides are as follows:

// Scalar <-- function(psi,i\_0,i\_1,....,i\_N)
inline T operator()(const std::size\_t\& ps\_index,
parameter\_list\_t parameters)
{
...
}

// String <-- function(psi,i\_0,i\_1,....,i\_N)
inline T operator()(const std::size\_t\& ps\_index,
std::string\& result,
parameter\_list\_t parameters)
{
...
}


When the function  operator is invoked  the 'ps\_index' parameter  will
have as its value the index of the parameter sequence that matches the
specific invocation. This way complex and time consuming type checking
conditions need not  be executed in  the function itself  but rather a
simple and efficient  dispatch to a  specific implementation for  that
particular parameter sequence can be performed.

template <typename T>
struct roo : public exprtk::igeneric\_function<T>
{
typedef typename exprtk::igeneric\_function<T>::parameter\_list\_t
parameter\_list\_t;

moo()
: exprtk::igeneric\_function<T>("SVTT|SS|TTV|S?V*S")
{}

inline T operator()(const std::size\_t\& ps\_index,
parameter\_list\_t parameters)
{
	...
}
};


In the example above there are four distinct parameter sequences  that
can be processed  by the generic  function 'roo'. Any  other parameter
sequences will cause a compilation error. The four valid sequences are
as follows:

Sequence-0    Sequence-1    Sequence-2    Sequence-3
'SVTT'         'SS'          'TTV'       'S?V*S'
(1) String    (1) String    (1) Scalar    (1) String
(2) Vector    (2) String    (2) Scalar    (2) Any Type
(3) Scalar                  (3) Vector    (3) One or more Vectors
(4) Scalar                                (4) String


(5) function\_compositor
The function  compositor is  a factory  that allows  one to define and
construct a function using ExprTk syntax. The functions are limited to
returning a single scalar value and consuming up to six parameters  as
input.

All composited functions are registered with a symbol table,  allowing
them to call other functions that have been registered with the symbol
table instance. Furthermore the  functions can be recursive  in nature
due to the inherent  function prototype forwarding that  occurs during
construction.  The following  example  defines, by using two different
methods, composited functions and implicitly registering the functions
with the denoted symbol table.

typedef exprtk::symbol\_table<T>         symbol\_table\_t;
typedef exprtk::function\_compositor<T>    compositor\_t;
typedef typename compositor\_t::function     function\_t;

symbol\_table\_t symbol\_table;

compositor\_t compositor(symbol\_table);

// define function koo0(v1,v2) { ... }
compositor
.add(
function\_t(
"koo0",
" 1 + cos(v1 * v2) / 3;",
"v1","v2"));

// define function koo1(x,y,z) { ... }
compositor
.add(function\_t()
.name("koo1")
.var("x").var("y").var("z")
.expression("1 + cos(x * y) / z;"));


(6) Using Functions In Expressions
For the above denoted custom and composited functions to be used in an
expression, an instance of each function needs to be registered with a
symbol\_table that  has been  associated with  the expression instance.
The following demonstrates how all the pieces are put together:

typedef exprtk::symbol\_table<double>      symbol\_table\_t;
typedef exprtk::expression<double>          expression\_t;
typedef exprtk::parser<double>                  parser\_t;
typedef exprtk::function\_compositor<double> compositor\_t;
typedef typename compositor\_t::function       function\_t;

foo<double> f;
boo<double> b;
too<double> t;
toupper<double> tu;

symbol\_table\_t symbol\_table;
compositor\_t   compositor(symbol\_table);

symbol\_table.add\_function("foo",f);
symbol\_table.add\_function("boo",b);
symbol\_table.add\_function("too",t);

symbol\_table.add\_function("toupper",
tu,
symbol\_table\_t::e\_ft\_strfunc);

compositor
.add(function\_t()
.name("koo")
.var("v1")
.var("v2")
.expression("1 + cos(v1 * v2) / 3;"));

expression\_t expression;
expression.register\_symbol\_table(symbol\_table);

std::string expression\_str =
" if (foo(1,2,3) + boo(1) > boo(1/2,2/3,3/4,4/5)) "
"   koo(3,4);                                     "
" else                                            "
"   too(2 * v1 + v2 / 3, 'abcdef'[2:4], 3.3);     "
"                                                 ";

parser\_t parser;
parser.compile(expression\_str,expression);

expression.value();


(7) Function Side-Effects
All function calls are assumed  to have side-effects by default.  This
assumption implicitly disables constant folding optimisations when all
parameters being passed to the function are deduced as being constants
at compile time.

If it is certain that the function being registered does not have  any
side effects and can  be correctly constant folded  where appropriate,
then during the construction of the function the side-effect trait  of
the function can be disabled.

template <typename T>
struct foo : public exprtk::ifunction<T>
{
foo() : exprtk::ifunction<T>(3)
{
	exprtk::disable\_has\_side\_effects(*this);
}

T operator()(const T\& v1, const T\& v2, const T\& v3)
{ ... }
};


(8) Zero Parameter Functions
When  either  an  ifunction,  ivararg\_function  or   igeneric\_function
derived type is defined with zero number of parameters, there are  two
calling  conventions  within  expressions  that  are  allowed.  For  a
function named 'foo' with zero input parameters the calling styles are
as follows:

(1)  x + sin(foo()- 2) / y
(2)  x + sin(foo  - 2) / y


By default the  zero parameter trait  is disabled. In  order to enable
it, a process similar to that of  enabling of the side effect trait is
carried out:

template <typename T>
struct foo : public exprtk::ivararg\_function<T>
{
foo()
{
	exprtk::enable\_zero\_parameters(*this);
}

inline T operator()(const std::vector<T>\& arglist)
{ ... }
};


Note: For  the igeneric\_function  type, there  also needs  to be a 'Z'
parameter sequence  defined in order for the  zero parameter  trait to
properly take effect otherwise a compilation error will occur.


(9) Free Functions
The ExprTk symbol  table supports the  registration of free  functions
and lambdas  (anonymous functors)  for use  in expressions.  The basic
requirements  are similar  to those  found in  ifunction derived  user
defined  functions. This  includes  support  for free  functions using
anywhere from zero up to fifteen input parameters of scalar type, with
a return type that is also scalar. Furthermore such functions will  by
default be assumed to have side-effects and hence will not participate
in constant folding optimisations.

In the following  example, a two  input parameter free  function named
'compute1', and a three  input parameter lambda named  'compute2' will
be registered with the given symbol\_table instance:


double compute1(double v0, double v1)
{
return 2.0 * v0 + v1 / 3.0;
}

.
.
.

typedef exprtk::symbol\_table<double> symbol\_table\_t;

symbol\_table\_t symbol\_table;

symbol\_table.add\_function("compute1", compute1);

symbol\_table.add\_function(
"compute2",
[](double v0, double v1, double v2) -> double
{ return v0 / v1 + v2; });

~~~~~~~~~~~~~~~~~~~~~~~~~~~~~~~~~~~~~~~~~~~~~~~~~~~~~~~~~~

[SECTION 16 - EXPRESSION DEPENDENTS]
Any  expression  that  is  not  a  literal  (aka  constant)  will have
dependencies. The types of  'dependencies' an expression can  have are
as follows:

(a) Variables
(b) Vectors
(c) Strings
(d) Functions
(e) Assignments


In  the  following  example the  denoted  expression  has its  various
dependencies listed:

z := abs(x + sin(2 * pi / y))

(a) Variables:   x, y, z and pi
(b) Functions:   abs, sin
(c) Assignments: z


ExprTk allows for  the derivation of  expression dependencies via  the
'dependent\_entity\_collector'  (DEC).  When  activated  either  through
'compile\_options' at the construction  of the parser or  through calls
to enabler methods just prior to compilation, the DEC will proceed  to
collect any  of the  relevant types  that are  encountered during  the
parsing  phase.   Once  the   compilation  process   has  successfully
completed, the  caller can  then obtain  a list  of symbols  and their
associated types from the DEC.

The kinds of  questions one can  ask regarding the  dependent entities
within an expression are as follows:

* What user defined variables, vectors or strings are used?
* What functions or custom user functions are used?
* Which variables, vectors or strings have values assigned to them?


The following example demonstrates usage of the DEC in determining the
dependents of the given expression:

typedef typename parser\_t::
dependent\_entity\_collector::symbol\_t symbol\_t;

std::string expression\_string =
"z := abs(x + sin(2 * pi / y))";

T x,y,z;

parser\_t parser;
symbol\_table\_t symbol\_table;

symbol\_table.add\_variable("x",x);
symbol\_table.add\_variable("y",y);
symbol\_table.add\_variable("z",z);

expression\_t expression;
expression.register\_symbol\_table(symbol\_table);

//Collect only variable and function symbols
parser.dec().collect\_variables() = true;
parser.dec().collect\_functions() = true;

if (!parser.compile(expression\_string,expression))
{
// error....
}

std::deque<symbol\_t> symbol\_list;

parser.dec().symbols(symbol\_list);

for (std::size\_t i = 0; i < symbol\_list.size(); ++i)
{
symbol\_t\& symbol = symbol\_list[i];

switch (symbol.second)
{
	case parser\_t::e\_st\_variable : ... break;
	case parser\_t::e\_st\_vector   : ... break;
	case parser\_t::e\_st\_string   : ... break;
	case parser\_t::e\_st\_function : ... break;
}
}


Note: The 'symbol\_t' type is a std::pair comprising of the symbol name
(std::string) and the associated type of the symbol as denoted by  the
cases in the switch statement.

Having  particular  symbols  (variable  or  function)  present  in  an
expression is one form of dependency. Another and just as  interesting
and important type of  dependency is that of  assignments. Assignments
are the set of dependent symbols that 'may' have their values modified
within an expression. The following are example expressions and  their
associated assignments:

Assignments   Expression
(1) x             x := y + z
(2) x, y          x += y += z
(3) x, y, z       x := y += sin(z := w + 2)
(4) w, z          if (x > y, z := x + 2, w := 'A String')
(5) None          x + y + z


Note: In expression 4, both variables 'w' and 'z' are denoted as being
assignments even though only one of  them can ever be modified at  the
time of evaluation. Furthermore the determination of which of the  two
variables the  modification will  occur upon  can only  be known  with
certainty at evaluation time and not beforehand, hence both are listed
as being candidates for assignment.

The following builds upon the previous example demonstrating the usage
of the DEC in determining the 'assignments' of the given expression:

//Collect assignments
parser.dec().collect\_assignments() = true;

if (!parser.compile(expression\_string,expression))
{
// error....
}

std::deque<symbol\_t> symbol\_list;

parser.dec().assignment\_symbols(symbol\_list);

for (std::size\_t i = 0; i < symbol\_list.size(); ++i)
{
symbol\_t\& symbol = symbol\_list[i];

switch (symbol.second)
{
	case parser\_t::e\_st\_variable : ... break;
	case parser\_t::e\_st\_vector   : ... break;
	case parser\_t::e\_st\_string   : ... break;
}
}


Note: The assignments will only consist of variable types and as  such
will not contain symbols denoting functions.

~~~~~~~~~~~~~~~~~~~~~~~~~~~~~~~~~~~~~~~~~~~~~~~~~~~~~~~~~~

[SECTION 17 - HIERARCHIES OF SYMBOL TABLES]
Most situations will only require a single symbol\_table instance to be
associated with a given expression instance.

However as an expression can have more than one symbol table  instance
associated  with  itself,  when  building  more  complex  systems that
utilise many expressions  where each can  in turn utilise  one or more
variables  from  a  large set  of  potential  variables, functions  or
constants, it becomes evident  that grouping variables into  layers of
symbol\_tables will simplify and streamline the overall process.

A suggested hierarchy of symbol tables is as follows:

(a) Global constant value symbol table
(b) Global non side-effect functions symbol table
(c) Global variable symbol table
(d) Expression specific variable symbol table


(a) Global constant value symbol table
This symbol table will  contain constant variables denoting  immutable
values. These variables can be made available to all expressions,  and
in turn expressions  will assume the  values themselves will  never be
modified for the  duration of the  process run-time. Examples  of such
variables are:

(1) pi or e
(2) speed\_of\_light
(3) avogadro\_number
(4) num\_cpus


(b) Global non side-effect functions symbol table
This symbol table will contain  only user defined functions that  will
not incur any side-effects that are visible to any of the  expressions
that invoke  them. These  functions will  be thread-safe  or threading
invariant and will not maintain any form of state between invocations.
Examples of such functions are:

(1) calc\_volume\_of\_sphere(r)
(2) distance(x0,y0,x1,y1)


(c) Global variable symbol table
This symbol table  will contain variables  that will be  accessible to
all associated expressions  and will not  be specific or  exclusive to
any one expression. This variant  differs from (a) in that  the values
of the  variables can  change (or  be updated)  between evaluations of
expressions  -   but  through   properly  scheduled   evaluations  are
guaranteed  to never  change during  the evaluation  of any  dependent
expressions. Furthermore it  is assumed that  these variables will  be
used in a  read-only context and  that no expressions  will attempt to
modify these variables via assignments or other means.

(1) price\_of\_stock\_xyz
(2) outside\_temperature or inside\_temperature
(3) fuel\_in\_tank
(4) num\_customers\_in\_store
(5) num\_items\_on\_shelf


(d) Expression specific variable symbol table
This  symbol\_table  is  the most  common form,  and is  used to  store
variables that are specific and exclusive to a particular  expression.
That is to say references  to variables in this symbol\_table  will not
be  part of  another expression.  Though it  may be  possible to  have
expressions that  contain the  variables with  the same  name, in that
case those variables will be distinctly different. Which would mean if
a particular  expression were  to be  compiled twice,  each expression
would have its  own unique symbol\_table  which in turn  would have its
own instances of those variables. Examples of such variables could be:

(1) x or y
(2) customer\_name


The  following is  a  diagram  depicting the  possible version  of the
denoted symbol table hierarchies. In the diagram there are two  unique
expressions, each of  which have a  reference to the  Global constant,
functions and variables symbol tables and an exclusive reference to  a
local symbol table.

+-------------------------+    +-------------------------+
|     Global Constants    |    |     Global Functions    |
|       Symbol Table      |    |       Symbol Table      |
+----o--o-----------------+    +--------------------o----+
|  |                                           |
|  |                                           +-------+
|  +------------------->----------------------------+  |
|         +----------------------------+            |  |
|         |      Global Variables      |            |  |
|  +------o        Symbol Table        o-----+      |  V
|  |      +----------------------------+     |      |  |
|  |                                         |      |  |
|  | +----------------+   +----------------+ |      |  |
|  | | Symbol Table 0 |   | Symbol Table 1 | |      V  |
|  | +--o-------------+   +--o-------------+ |      |  |
|  |    |                    |               |      |  |
|  |    |                    |               |      |  |
+--V--V----V---------+        +-V---------------V--+   |  |
|    Expression 0    |        |    Expression 1    |<--+--+
|  '2 * sin(x) - y'  |        |  'k + abs(x - y)'  |
+--------------------+        +--------------------+


Bringing  all of  the above  together, in  the following  example the
hierarchy  of  symbol  tables  are instantiated  and  initialised. An
expression that makes use of various elements of each symbol table is
then compiled and later on evaluated:

typedef exprtk::symbol\_table<double> symbol\_table\_t;
typedef exprtk::expression<double>     expression\_t;

// Setup global constants symbol table
symbol\_table\_t glbl\_const\_symbol\_table;
glbl\_const\_symbtab.add\_constants(); // pi, epsilon and inf
glbl\_const\_symbtab.add\_constant("speed\_of\_light",299e6);
glbl\_const\_symbtab.add\_constant("avogadro\_number",6e23);

// Setup global function symbol table
symbol\_table\_t glbl\_funcs\_symbol\_table;
glbl\_func\_symbtab.add\_function('distance',distance);
glbl\_func\_symbtab.add\_function('calc\_spherevol',calc\_sphrvol);

......

// Setup global variable symbol table
symbol\_table\_t glbl\_variable\_symbol\_table;
glbl\_variable\_symbtab.add\_variable('temp\_outside',thermo.outside);
glbl\_variable\_symbtab.add\_variable('temp\_inside' ,thermo.inside );
glbl\_variable\_symbtab.add\_variable('num\_cstmrs',store.num\_cstmrs);

......

double x,y,z;

// Setup expression specific symbol table
symbol\_table\_t symbol\_table;
symbol\_table.add\_variable('x',x);
symbol\_table.add\_variable('y',y);
symbol\_table.add\_variable('z',z);

expression\_t expression;

// Register the various symbol tables
expression
.register\_symbol\_table(symbol\_table);

expression
.register\_symbol\_table(glbl\_funcs\_symbol\_table);

expression
.register\_symbol\_table(glbl\_const\_symbol\_table);

expression
.register\_symbol\_table(glbl\_variable\_symbol\_table);

std::string expression\_str =
"abs(temp\_inside - temp\_outside) + 2 * speed\_of\_light / x";

parser\_t parser;
parser.compile(expression\_str,expression);

......

while (keep\_evaluating)
{
....

T result = expression.value();

....
}

~~~~~~~~~~~~~~~~~~~~~~~~~~~~~~~~~~~~~~~~~~~~~~~~~~~~~~~~~~

[SECTION 18 - UNKNOWN UNKNOWNS]
In this section  we will discuss  the process of  handling expressions
with a mix of known and unknown variables. Initially a discussion into
the types of expressions that exist will be provided, then a series of
possible solutions will be presented for each scenario.

When parsing an expression, there  may be situations where one  is not
fully  aware  of what  if  any variables  will  be used  prior  to the
expression being compiled.

This can become problematic, as in the default scenario it is  assumed
the symbol\_table that is registered with the expression instance  will
already  posses  the  externally  available  variables,  functions and
constants needed during the compilation of the expression.

In the event there are symbols in the expression that can't be  mapped
to   either   a  reserved   word,   or  located   in   the  associated
symbol\_table(s), an "Undefined  symbol" error will  be raised and  the
compilation process will fail.

The numerous  scenarios that  can occur  when compiling  an expression
with ExprTk generally fall into one of the following three categories:

(a) No external variables
(b) Predetermined set of external variables
(c) Unknown set of variables


(a) No external variables
These  are  expressions that  contain  no external  variables  but may
contain  local  variables.  As  local  variables  cannot  be  accessed
externally from the  expression, it is  assumed that such  expressions
will not have  a need for  a symbol\_table and  furthermore expressions
which  don't  make  use of  functions that  have side-effects  will be
evaluated completely at  compile time resulting  in a constant  return
value. The following are examples of such expressions:

(1) 1 + 2
(2) var x := 3; 2 * x - 3
(3) var x := 3; var y := abs(x - 8); x - y / 7


(b) Predetermined set of external variables
These  are  expressions  that are  comprised  of  externally available
variables  and functions  and will  only compile  successfully if  the
symbols that  correspond to  the variables  and functions  are already
defined in their associated symbol\_table(s).  This is by far the  most
common scenario when using ExprTk.

As an example, one may have three external variables: x, y and z which
have been registered with  the associated symbol\_table, and  will then
need to compile  and evaluate expressions  comprised of any  subset of
these  three  variables. The  following  are a  few  examples of  such
expressions:

(1) 1 + x
(2) x / y
(3) 2 * x * y / z


In  this  scenario   one  can  use   the  'dependent\_entity\_collector'
component as described in [Section  16] to further determine which  of
the registered variables were  actually used in the  given expression.
As  an example  once the  set of  utilised  variables  are known,  any
further 'attention'  can be  restricted to  only those  variables when
evaluating the expression. This can be quite useful when dealing  with
expressions that can draw from a set of hundreds or even thousands  of
variables.


(c) Unknown set of variables
These are  expressions that  are comprised  of symbols  other than the
standard ExprTk reserved words or what has been registered with  their
associated symbol\_table, and will normally fail compilation due to the
associated symbol\_table not having a  reference to them. As such  this
scenario can be  seen as a  combination of scenario  B, where one  may
have a symbol\_table with registered variables, but would also like  to
handle  the  situation  of  variables  that  aren't  present  in  said
symbol\_table.

When dealing with expressions of category (c), one must perform all of
the following:

(1) Determine the variables used in the expression
(2) Populate a symbol\_table(s) with the entities from (1)
(3) Compile the expression
(4) Provide a means by which the entities from (1) can be modified


Depending on the nature of processing,  steps (1) and (2) can be  done
either independently of each other or combined into one. The following
example  will  initially  look  at  solving  the  problem  of  unknown
variables with the  latter method using  the 'unknown\_symbol\_resolver'
component.

typedef exprtk::symbol\_table<T> symbol\_table\_t;
typedef exprtk::expression<T>     expression\_t;
typedef exprtk::parser<T>             parser\_t;

symbol\_table\_t unknown\_var\_symbol\_table;

symbol\_table\_t symbol\_table;
symbol\_table.add\_variable("x",x);
symbol\_table.add\_variable("y",y);

expression\_t expression;
expression.register\_symbol\_table(unknown\_var\_symbol\_table);
expression.register\_symbol\_table(symbol\_table);

parser\_t parser;
parser.enable\_unknown\_symbol\_resolver();

std::string expression\_str = "x + abs(y / 3k) * z + 2";

parser.compile(expression\_str,expression);


In the  example above,  the symbols  'k' and  'z' will  be treated  as
unknown symbols. The  parser in the  example is set  to handle unknown
symbols using the built-in default unknown\_symbol\_resolver (USR).  The
default  USR  will  automatically resolve  any  unknown  symbols as  a
variable (scalar type). The new variables will be added to the primary
symbol\_table,  which in  this case  is the  'unknown\_var\_symbol\_table'
instance.  Once  the  compilation  has  completed  successfully,   the
variables that were resolved  during compilation can be  accessed from
the   primary   symbol\_table   using   the   'get\_variable\_list'   and
'variable\_ref' methods and then if needed can be modified  accordingly
after which the expression itself can be evaluated.

std::vector<std::string> variable\_list;

unknown\_var\_symbol\_table.get\_variable\_list(variable\_list);

for (auto\& var\_name : variable\_list)
{
T\& v = symbol\_table.variable\_ref(var\_name);

v = ...;
}

...

expression.value();


Note:  As  previously  mentioned the  default  USR  will automatically
assume any unknown symbol to be a valid scalar variable, and will then
proceed to add said symbol  as a variable to the  primary symbol\_table
of the associated expression during the compilation process. However a
problem that may arise, is  that expressions that are parsed  with the
USR enabled,  but contain  'typos' or  otherwise syntactic  errors may
inadvertently compile successfully due to the simplistic nature of the
default USR. The following are some example expressions:

(1) 1 + abz(x + 1)
(2) sine(y / 2) - coz(3x)


The two  expressions above contain misspelt  symbols (abz,  sine, coz)
which if implied  multiplications and default  USR are enabled  during
compilation will result in them being assumed to be valid 'variables',
which obviously is  not the intended  outcome by the  user. A possible
solution to this  problem is for  one to implement  their own specific
USR that will perform a user defined business logic in determining  if
an encountered unknown symbol should be treated as a variable or if it
should raise a compilation error. The following example demonstrates a
simple user defined USR:

typedef exprtk::symbol\_table<T> symbol\_table\_t;
typedef exprtk::expression<T>     expression\_t;
typedef exprtk::parser<T>             parser\_t;

template <typename T>
struct my\_usr : public parser\_t::unknown\_symbol\_resolver
{
typedef typename parser\_t::unknown\_symbol\_resolver usr\_t;

bool process(const std::string\& unknown\_symbol,
typename usr\_t::usr\_symbol\_type\& st,
T\& default\_value,
std::string\& error\_message)
{
	if (0 != unknown\_symbol.find("var\_"))
	{
		error\_message = "Invalid symbol: " + unknown\_symbol;
		return false;
	}
	
	st = usr\_t::e\_usr\_variable\_type;
	default\_value = T(123.123);
	
	return true;
}
};

...

symbol\_table\_t unknown\_var\_symbol\_table;

symbol\_table\_t symbol\_table;
symbol\_table.add\_variable("x",x);
symbol\_table.add\_variable("y",y);

expression\_t expression;
expression.register\_symbol\_table(unknown\_var\_symbol\_table);
expression.register\_symbol\_table(symbol\_table);

my\_usr<T> musr;

parser\_t parser;
parser.enable\_unknown\_symbol\_resolver(\&musr);

std::string expression\_str = "var\_x + abs(var\_y - 3) * var\_z";

parser.compile(expression\_str,expression);


In  the  example  above,  a user  specified  USR  is  defined, and  is
registered with the parser  enabling the USR functionality.  Then when
an unknown symbol is  encountered during the compilation  process, the
USR's process method will be invoked. The USR in the example will only
'accept' unknown symbols that have  a prefix of 'var\_' as  being valid
variables,  all other  unknown symbols  will result  in a  compilation
error being raised.

In the example above  the callback of the  USR that is invoked  during
the unknown symbol resolution process only allows for scalar variables
to be defined and resolved -  as that is the simplest and  most common
form.

There  is  also  an  extended version  of  the  callback  that can  be
overridden that will allow for  more control and choice over  the type
of symbol being  resolved. The following  is an example  definition of
said extended callback:

template <typename T>
struct my\_usr : public parser\_t::unknown\_symbol\_resolver
{
typedef typename parser\_t::unknown\_symbol\_resolver usr\_t;

my\_usr()
: usr\_t(usr\_t::e\_usrmode\_extended)
{}

virtual bool process(const std::string\& unknown\_symbol,
symbol\_table\_t\&      symbol\_table,
std::string\&        error\_message)
{
	bool result = false;
	
	if (0 == unknown\_symbol.find("var\_"))
	{
		// Default value of zero
		result = symbol\_table.create\_variable(unknown\_symbol,0);
		
		if (!result)
		{
			error\_message = "Failed to create variable...";
		}
	}
	else if (0 == unknown\_symbol.find("str\_"))
	{
		// Default value of empty string
		result = symbol\_table.create\_stringvar(unknown\_symbol,"");
		
		if (!result)
		{
			error\_message = "Failed to create string variable...";
		}
	}
	else
	error\_message = "Indeterminable symbol type.";
	
	return result;
}
};


In the  example above,  the USR  callback when  invoked will  pass the
primary symbol table associated with the expression being parsed.  The
symbol  resolution  business  logic  can  then  determine  under  what
conditions  a  symbol will  be  resolved including  its  type (scalar,
string, vector etc) and default value. When the callback  successfully
returns  the  symbol  parsing and  resolution  process  will again  be
executed by the parser. The idea here is that given the primary symbol
table will now have the previously detected unknown symbol registered,
it will be correctly resolved  and the general parsing processing  can
then resume as per normal.

Note: In order to have the USR's extended mode callback be invoked  It
is  necessary to  pass the  e\_usrmode\_extended enum  value during  the
constructor of the user defined USR.

Note: The primary symbol table  for an expression is the  first symbol
table to be registered with that instance of the expression.

~~~~~~~~~~~~~~~~~~~~~~~~~~~~~~~~~~~~~~~~~~~~~~~~~~~~~~~~~~

[SECTION 19 - ENABLING \& DISABLING FEATURES]
The parser can be configured via its settings instance to either allow
or  disallow certain  features that  are available  within the  ExprTk
grammar. The features fall  into one  of the following six categories:

(1) Base Functions
(2) Control Flow Structures
(3) Logical Operators
(4) Arithmetic Operators
(5) Inequality Operators
(6) Assignment Operators


(1) Base Functions
The list of available base functions is as follows:

abs, acos, acosh, asin,  asinh, atan, atanh, atan2,  avg, ceil,
clamp,  cos,  cosh, cot,  csc,  equal, erf,  erfc,  exp, expm1,
floor,  frac,  hypot,  iclamp, like,  log,  log10,  log2, logn,
log1p, mand, max, min, mod,  mor, mul, ncdf, pow, root,  round,
roundn, sec, sgn, sin, sinc, sinh, sqrt, sum, swap, tan,  tanh,
trunc, not\_equal, inrange, deg2grad, deg2rad, rad2deg, grad2deg


The above mentioned base functions  can be either enabled or  disabled
'all' at once, as is demonstrated below:

parser\_t parser;
expression\_t expression;

parser.settings().disable\_all\_base\_functions();

parser
.compile("2 * abs(2 - 3)",expression); // compilation failure

parser.settings().enable\_all\_base\_functions();

parser
.compile("2 * abs(2 - 3)",expression); // compilation success


One can also enable or disable specific base functions. The  following
example  demonstrates  the disabling  of  the trigonometric  functions
'sin' and 'cos':

parser\_t parser;
expression\_t expression;

parser.settings()
.disable\_base\_function(settings\_t::e\_bf\_sin)
.disable\_base\_function(settings\_t::e\_bf\_cos);

parser
.compile("(sin(x) / cos(x)) == tan(x)",expression); // failure

parser.settings()
.enable\_base\_function(settings\_t::e\_bf\_sin)
.enable\_base\_function(settings\_t::e\_bf\_cos);

parser
.compile("(sin(x) / cos(x)) == tan(x)",expression); // success


(2) Control Flow Structures
The list of available control flow structures is as follows:

(a) If or If-Else
(b) Switch statement
(c) For Loop
(d) While Loop
(e) Repeat Loop


The  above  mentioned  control flow structures  can be  either enabled
or disabled 'all' at once, as is demonstrated below:

parser\_t parser;
expression\_t expression;

std::string program =
" var x := 0;                      "
" for (var i := 0; i < 10; i += 1) "
" {                                "
"   x += i;                        "
" }                                ";

parser.settings().disable\_all\_control\_structures();

parser
.compile(program,expression); // compilation failure

parser.settings().enable\_all\_control\_structures();

parser
.compile(program,expression); // compilation success


One can also enable or  disable specific control flow structures.  The
following example demonstrates the  disabling of the for-loop  control
flow structure:

parser\_t parser;
expression\_t expression;

std::string program =
" var x := 0;                      "
" for (var i := 0; i < 10; i += 1) "
" {                                "
"   x += i;                        "
" }                                ";

parser.settings()
.disable\_control\_structure(settings\_t::e\_ctrl\_for\_loop);

parser
.compile(program,expression); // failure

parser.settings()
.enable\_control\_structure(settings\_t::e\_ctrl\_for\_loop);

parser
.compile(program,expression); // success


(3) Logical Operators
The list of available logical operators is as follows:

and, nand, nor, not, or, xnor, xor, \&, |


The  above  mentioned  logical  operators  can  be  either  enabled or
disabled 'all' at once, as is demonstrated below:

parser\_t parser;
expression\_t expression;

parser.settings().disable\_all\_logic\_ops();

parser
.compile("1 or not(0 and 1)",expression); // compilation failure

parser.settings().enable\_all\_logic\_ops();

parser
.compile("1 or not(0 and 1)",expression); // compilation success


One  can  also  enable  or  disable  specific  logical  operators. The
following  example  demonstrates  the disabling  of the  'and' logical
operator:

parser\_t parser;
expression\_t expression;

parser.settings()
.disable\_logic\_operation(settings\_t::e\_logic\_and);

parser
.compile("1 or not(0 and 1)",expression); // failure

parser.settings()
.enable\_logic\_operation(settings\_t::e\_logic\_and);

parser
.compile("1 or not(0 and 1)",expression); // success


(4) Arithmetic Operators
The list of available arithmetic operators is as follows:

+, -, *, /, \%, \^\\


The  above mentioned  arithmetic operators  can be  either enabled  or
disabled 'all' at once, as is demonstrated below:

parser\_t parser;
expression\_t expression;

parser.settings().disable\_all\_arithmetic\_ops();

parser
.compile("1 + 2 / 3",expression); // compilation failure

parser.settings().enable\_all\_arithmetic\_ops();

parser
.compile("1 + 2 / 3",expression); // compilation success


One  can also  enable or  disable specific  arithmetic operators.  The
following  example  demonstrates  the disabling  of  the  addition '+'
arithmetic operator:

parser\_t parser;
expression\_t expression;

parser.settings()
.disable\_arithmetic\_operation(settings\_t::e\_arith\_add);

parser
.compile("1 + 2 / 3",expression); // failure

parser.settings()
.enable\_arithmetic\_operation(settings\_t::e\_arith\_add);

parser
.compile("1 + 2 / 3",expression); // success


(5) Inequality Operators
The list of available inequality operators is as follows:

<, <=, >, >=, ==, =, != <>


The  above mentioned  inequality operators  can be  either enabled  or
disabled 'all' at once, as is demonstrated below:

parser\_t parser;
expression\_t expression;

parser.settings().disable\_all\_inequality\_ops();

parser
.compile("1 < 3",expression); // compilation failure

parser.settings().enable\_all\_inequality\_ops();

parser
.compile("1 < 3",expression); // compilation success


One  can also  enable or  disable specific  inequality operators.  The
following  example demonstrates  the disabling  of  the  less-than '<'
inequality operator:

parser\_t parser;
expression\_t expression;

parser.settings()
.disable\_inequality\_operation(settings\_t::e\_ineq\_lt);

parser
.compile("1 < 3",expression); // failure

parser.settings()
.enable\_inequality\_operation(settings\_t::e\_ineq\_lt);

parser
.compile("1 < 3",expression); // success


(6) Assignment Operators
The list of available assignment operators is as follows:

:=, +=, -=, *=, /=, \%=


The  above mentioned  assignment operators  can be  either enabled  or
disabled 'all' at once, as is demonstrated below:

parser\_t parser;
expression\_t expression;
symbol\_table\_t symbol\_table;

T x = T(0);

symbol\_table.add\_variable("x",x);

expression.register\_symbol\_table(symbol\_table);

parser.settings().disable\_all\_assignment\_ops();

parser
.compile("x := 3",expression); // compilation failure

parser.settings().enable\_all\_assignment\_ops();

parser
.compile("x := 3",expression); // compilation success


One  can also  enable or  disable specific  assignment operators.  The
following  example demonstrates  the  disabling  of the  '+=' addition
assignment operator:

parser\_t parser;
expression\_t expression;
symbol\_table\_t symbol\_table;

T x = T(0);

symbol\_table.add\_variable("x",x);

expression.register\_symbol\_table(symbol\_table);

parser.settings()
.disable\_assignment\_operation(settings\_t::e\_assign\_addass);

parser
.compile("x += 3",expression); // failure

parser.settings()
.enable\_assignment\_operation(settings\_t::e\_assign\_addass);

parser
.compile("x += 3",expression); // success


Note: In the event of a base function being disabled, one can redefine
the  base  function  using  the  standard  custom  function definition
process. In the  following example the 'sin' function is disabled then
redefined as a function taking degree input.

template <typename T>
struct sine\_deg : public exprtk::ifunction<T>
{
sine\_deg() : exprtk::ifunction<T>(1) {}

inline T operator()(const T\& v)
{
	const T pi = exprtk::details::numeric::constant::pi;
	return std::sin((v * T(pi)) / T(180));
}
};

...

typedef exprtk::symbol\_table<T> symbol\_table\_t;
typedef exprtk::expression<T>     expression\_t;
typedef exprtk::parser<T>             parser\_t;

typedef typename parser\_t::settings\_store settings\_t;

sine\_deg<T> sine;

symbol\_table.add\_reserved\_function("sin",sine);

expression\_t expression;

expression.register\_symbol\_table(symbol\_table);

parser\_t parser;

parser.settings()
.disable\_base\_function(settings\_t::e\_bf\_sin);

parser.compile("1 + sin(30)",expression);


In the example above, the custom 'sin' function is registered with the
symbol\_table using the method 'add\_reserved\_function'. This is done so
as to bypass the checks for reserved words that are carried out on the
provided symbol names when calling the standard 'add\_function' method.
Normally if  a user  specified symbol  name conflicts  with any of the
ExprTk reserved words, the add\_function call will fail.

~~~~~~~~~~~~~~~~~~~~~~~~~~~~~~~~~~~~~~~~~~~~~~~~~~~~~~~~~~

[SECTION 20 - EXPRESSION RETURN VALUES]
ExprTk expressions can return immediately from any point by  utilizing
the return call. Furthermore the  return call can be used  to transfer
out multiple return values from within the expression.

If an expression evaluation exits using a return point, the result  of
the call to the 'value' method will be NaN, and it's expected that the
return values will be available from the results\_context.

In  the  following example  there  are  three  return  points  in  the
expression.  If  neither  of  the  return  points  are  hit,  then the
expression will return normally.

std::string expression\_string =
" if (x < y)                                   "
"   return [x + 1,'return-call 1'];            "
" else if (x > y)                              "
"   return [y / 2, y + 1, 'return-call 2'];    "
" else if (equal(x,y))                         "
"   x + y;                                     "
" return [x, y, x + y, x - y, 'return-call 3'] ";

typedef exprtk::symbol\_table<double> symbol\_table\_t;
typedef exprtk::expression<double>     expression\_t;
typedef exprtk::parser<double>             parser\_t;

symbol\_table\_t symbol\_table;
expression\_t   expression;
parser\_t       parser;

double x = 0;
double y = 0;

symbol\_table.add\_variable("x",x);
symbol\_table.add\_variable("y",y);

expression.register\_symbol\_table(symbol\_table);

parser.compile(expression\_string,expression);

T result = expression.value();

if (expression.results().count())
{
typedef exprtk::results\_context<T> results\_context\_t;
typedef typename results\_context\_t::type\_store\_t type\_t;
typedef typename type\_t::scalar\_view scalar\_t;
typedef typename type\_t::vector\_view vector\_t;
typedef typename type\_t::string\_view string\_t;

const results\_context\_t\& results = expression.results();

for (std::size\_t i = 0; i < results.count(); ++i)
{
	type\_t t = results[i];
	
	switch (t.type)
	{
		case type\_t::e\_scalar : ...
		break;
		
		case type\_t::e\_vector : ...
		break;
		
		case type\_t::e\_string : ...
		break;
		
		default               : continue;
	}
}


Note: Processing of the return results is similar to that of the
generic function call parameters.

It is however recommended that if there is to be only a single flow of
execution  through  the  expression,  that  the  simpler  approach  of
registering external variables of appropriate type be used.

This method simply requires the variables that are to hold the various
results that are to be computed within the expression to be registered
with an associated symbol\_table  instance. Then within the  expression
itself  to  have  the result  variables  be  assigned the  appropriate
values.

typedef exprtk::symbol\_table<double> symbol\_table\_t;
typedef exprtk::expression<double>     expression\_t;
typedef exprtk::parser<double>             parser\_t;

std::string expression\_string =
" var x := 123.456;     "
" var s := 'ijk';       "
" result0 := x + 78.90; "
" result1 := s + '123'  ";

double      result0;
std::string result1;

symbol\_table\_t symbol\_table;
symbol\_table.add\_variable ("result0",result0);
symbol\_table.add\_stringvar("result1",result1);

expression\_t expression;
expression.register\_symbol\_table(symbol\_table);

parser\_t parser;
parser.compile(expression\_string,expression);

expression.value();

printf("Result0: \%15.5f\\n", result0        );
printf("Result1: \%s\\n"    , result1.c\_str());


In the example above, the expression will compute two results. As such
two result variables are defined to hold the values named result0  and
result1 respectively. The first is of scalar type (double), the second
is of  string type.  Once the  expression has  been evaluated, the two
variables will have been updated  with the new result values,  and can
then be further utilised from within the calling program.

~~~~~~~~~~~~~~~~~~~~~~~~~~~~~~~~~~~~~~~~~~~~~~~~~~~~~~~~~~

[SECTION 21 - COMPILATION ERRORS]
When attempting to compile  a malformed or otherwise  erroneous ExprTk
expression, the  compilation process  will result  in an  error, as is
indicated  by  the  'compile'  method  returning  a  false  value.   A
diagnostic indicating the first error encountered and its cause can be
obtained by  invoking the  'error' method,  as is  demonstrated in the
following example:

if (!parser.compile(expression\_string,expression))
{
	printf("Error: \%s\\n", parser.error().c\_str());
	return false;
}


Any error(s) resulting from a failed compilation will be stored in the
parser instance until the next time a compilation is performed. Before
then errors can be enumerated  in the order they occurred  by invoking
the 'get\_error' method which itself will return a 'parser\_error' type.
A parser\_error object will contain an error diagnostic, an error  mode
(or class), and the character position of the error in the  expression
string. The following example demonstrates the enumeration of error(s)
in the event of a failed compilation.

if (!parser.compile(expression\_string,expression))
{
	for (std::size\_t i = 0; i < parser.error\_count(); ++i)
	{
		typedef exprtk::parser\_error::type error\_t;
		
		error\_t error = parser.get\_error(i);
		
		printf("Error[\%02d] Position: \%02d Type: [\%14s] Msg: \%s\\n",
		i,
		error.token.position,
		exprtk::parser\_error::to\_str(error.mode).c\_str(),
		error.diagnostic.c\_str());
	}
	
	return false;
}


Assuming the  following expression '2 + (3 / log(1 + x))' which uses a
variable named 'x'  that has not been registered  with the appropriate
symbol\_table  instance and  is not  a locally  defined variable,  once
compiled the  above denoted post compilation error handling code shall
produce the following output:

Error: ERR184 - Undefined symbol: 'x'
Error[00] Pos:17 Type:[Syntax] Msg: ERR184 - Undefined symbol: 'x'


For  expressions  comprised  of  multiple  lines,  the  error position
provided in the  parser\_error object can  be converted into  a pair of
line and column numbers by invoking the 'update\_error' function as  is
demonstrated by the following example:

if (!parser.compile(program\_str,expression))
{
	for (std::size\_t i = 0; i < parser.error\_count(); ++i)
	{
		typedef exprtk::parser\_error::type error\_t;
		
		error\_t error = parser.get\_error(i);
		
		exprtk::parser\_error::update\_error(error,program\_str);
		
		printf("Error[\%02d] at line: \%d column: \%d\\n",
		i,
		error.line\_no,
		error.column\_no);
	}
	
	return false;
}


Note: There are five  distinct error modes in  ExprTk which denote the
class of an error. These classes are as follows:

(a) Syntax
(b) Token
(c) Numeric
(d) Symbol Table
(e) Lexer


(a) Syntax Errors
These are errors  related to invalid  syntax found within  the denoted
expression. Examples are invalid sequences of operators and variables,
incorrect number  of parameters  to functions,  invalid conditional or
loop structures and invalid use of keywords.

eg:  'for := sin(x,y,z) + 2 * equal > until[2 - x,3]'


(b) Token Errors
Errors in this class relate to  token level errors detected by one  or
more of the following checkers:

(1) Bracket Checker
(2) Numeric Checker
(3) Sequence Checker


(c) Numeric Errors
This class of  error is related  to conversion of  numeric values from
their  string form  to the  underlying numerical  type (float,  double
etc).

(d) Symbol Table Errors
This is the class of errors related to failures when interacting  with
the registered symbol\_table instance. Errors such as not being able to
find,  within  the  symbol\_table,  symbols  representing  variables or
functions, to being unable to create new variables in the symbol\_table
via the 'unknown symbol resolver' mechanism.

~~~~~~~~~~~~~~~~~~~~~~~~~~~~~~~~~~~~~~~~~~~~~~~~~~~~~~~~~~

[SECTION 22 - RUNTIME LIBRARY PACKAGES]
ExprTk   contains   a   set  of   simple   extensions,   that  provide
functionalities  beyond  basic numerical  calculations.  Currently the
available packages are:

+---+--------------------+-----------------------------------+
| \# |    Package Name    |          Namespace/Type           |
+---+--------------------+-----------------------------------+
| 1 | Basic I/O          | exprtk::rtl::io::package<T>       |
| 2 | File I/O           | exprtk::rtl::io::file::package<T> |
| 3 | Vector Operations  | exprtk::rtl::vecops::package<T>   |
+---+--------------------+-----------------------------------+


In order to make the  features of a specific package  available within
an  expression,  an instance  of  the package  must  be added  to  the
expression's associated  symbol table.  In the  following example, the
file I/O package is made available for the given expression:

typedef exprtk::symbol\_table<T> symbol\_table\_t;
typedef exprtk::expression<T>     expression\_t;
typedef exprtk::parser<T>             parser\_t;

exprtk::rtl::io::file::package<T> fileio\_package;

std::string expression\_string =
" var file\_name := 'file.txt';        "
" var stream    := null;              "
"                                     "
" stream := open(file\_name,'w');      "
"                                     "
" write(stream,'Hello world....\\n');  "
"                                     "
" close(stream);                      "
"                                     ";

symbol\_table\_t symbol\_table;
symbol\_table.add\_package(fileio\_package);

expression\_t expression;
expression.register\_symbol\_table(symbol\_table);

parser\_t parser;
parser.compile(expression\_string,expression);

expression.value();


(1) Basic I/O functions:

(a) print
(b) println

(2) File I/O functions:

(a) open    (b) close
(c) write   (d) read
(e) getline (f) eof

(3) Vector Operations functions:

(a) all\_true    (b) all\_false
(c) any\_true    (d) any\_false
(e) count       (f) copy
(g) rotate-left (h) rotate-right
(i) shift-left  (j) shift-right
(k) sort        (l) nth\_element
(m) iota        (n) sumk
(o) axpy        (p) axpby
(q) axpyz       (r) axpbyz
(s) axpbz       (t) dot
(u) dotk

~~~~~~~~~~~~~~~~~~~~~~~~~~~~~~~~~~~~~~~~~~~~~~~~~~~~~~~~~~

[SECTION 23 - HELPERS \& UTILS]
The  ExprTk library  provides a  series of  usage simplifications  via
helper routines that combine various processes into a single 'function
call'  making  certain  actions   easier  to  carry  out   though  not
necessarily in the most efficient way possible. A list of the routines
are as follows:

(a) collect\_variables
(b) collect\_functions
(c) compute
(d) integrate
(e) derivative
(f) second\_derivative
(g) third\_derivative


(a) collect\_variables
This function will collect all the variable symbols in a given  string
representation of an expression and  return them in an STL  compatible
sequence  data structure  (eg: std::vector,  dequeue etc)  specialised
upon a std::string type. If an error occurs during the parsing of  the
expression  then  the return  value  of the  function  will be  false,
otherwise it will be  true. An example use  of the given routine is as
follows:

std::string expression = "x + abs(y / z)";

std::vector<std::string> variable\_list;

if (exprtk::collect\_variables(expression, variable\_list))
{
	for (const auto\& var : variable\_list)
	{
		...
	}
}
else
printf("An error occured.");


(b) collect\_functions
This function will collect all the function symbols in a given  string
representation of an expression and  return them in an STL  compatible
sequence  data structure  (eg: std::vector,  dequeue etc)  specialised
upon a std::string type. If an error occurs during the parsing of  the
expression  then  the return  value  of the  function  will be  false,
otherwise it  will be true. An example  use of the given routine is as
follows:

std::string expression = "x + abs(y / cos(1 + z))";

std::deque<std::string> function\_list;

if (exprtk::collect\_functions(expression, function\_list))
{
	for (const auto\& func : function\_list)
	{
		...
	}
}
else
printf("An error occured.");


Note: When either the 'collect\_variables'  or 'collect\_functions' free
functions  return  true  -  that  does  not  necessarily  indicate the
expression itself is  valid. It is  still possible that  when compiled
the expression may have certain  'type' related errors - though  it is
highly likely  that no  semantic errors  will occur  if either  return
true.

Note: The  default interface  provided for  both the collect\_variables
and collect\_functions  free\_functions, assumes  that expressions  will
only be utilising  the ExprTk reserved  funnctions (eg: abs,  cos, min
etc). When user defined functions are  to be used in an expression,  a
symbol\_table  instance  containing  said functions  can  be  passed to
either routine, and  will be incorparated  during the compilation  and
Dependent Entity  Collection processes.  In the  following example,  a
user  defined  free  function   named  'foo'  is  registered   with  a
symbol\_table.  Finally  the   symbol\_table  instance  and   associated
expression string are passed to the exprtk::collect\_functions routine.

template <typename T>
T foo(T v)
{
	return std::abs(v + T(2)) / T(3);
}

......

exprtk::symbol\_table<T> sym\_tab;

symbol\_table.add\_function("foo",foo);

std::string expression = "x + foo(y / cos(1 + z))";

std::deque<std::string> function\_list;

if (exprtk::collect\_functions(expression, sym\_tab, function\_list))
{
	for (const auto\& func : function\_list)
	{
		...
	}
}
else
printf("An error occured.");


(c) compute
This free function  will compute the  value of an  expression from its
string form.  If an  invalid expression  is passed,  the result of the
function will be false indicating an error, otherwise the return value
will  be  true  indicating success.  The  compute  function has  three
overloads, the definitions of which are:

(1) No variables
(2) One variable called x
(3) Two variables called x and y
(3) Three variables called x, y and z


Example uses of  each of the  three overloads for  the compute routine
are as follows:

T result = T(0);

// No variables overload
std::string no\_vars = "abs(1 - (3 / pi)) * 5";

if (!exprtk::compute(no\_vars,result))
printf("Failed to compute: \%s",no\_vars.c\_str());
else
printf("Result: \%15.5f\\n",result);

// One variable 'x' overload
T x = 123.456;

std::string one\_var = "abs(x - (3 / pi)) * 5";

if (!exprtk::compute(one\_var, x, result))
printf("Failed to compute: \%s",one\_var.c\_str());
else
printf("Result: \%15.5f\\n",result);

// Two variables 'x' and 'y' overload
T y = 789.012;

std::string two\_var = "abs(x - (y / pi)) * 5";

if (!exprtk::compute(two\_var, x, y, result))
printf("Failed to compute: \%s",two\_var.c\_str());
else
printf("Result: \%15.5f\\n",result);

// Three variables 'x', 'y' and 'z' overload
T z = 345.678;

std::string three\_var = "abs(x - (y / pi)) * z";

if (!exprtk::compute(three\_var, x, y, z, result))
printf("Failed to compute: \%s",three\_var.c\_str());
else
printf("Result: \%15.5f\\n",result);


(d) integrate
This free function will attempt to perform a numerical integration  of
a single variable compiled expression over a specified range and  step
size. The numerical  integration is based  on the three  point form of
Simpson's rule. The  integrate function has  two overloads, where  the
variable of integration can  either be passed as  a reference or as  a
name in string form. Example usage of the function is as follows:

typedef exprtk::parser<T>             parser\_t;
typedef exprtk::expression<T>     expression\_t;
typedef exprtk::symbol\_table<T> symbol\_table\_t;

std::string expression\_string = "sqrt(1 - (x\^2))";

T x = T(0);

symbol\_table\_t symbol\_table;
symbol\_table.add\_variable("x",x);

expression\_t expression;
expression.register\_symbol\_table(symbol\_table);

parser\_t parser;
parser.compile(expression\_string,expression);

....

// Integrate in domain [-1,1] using a reference to x variable
T area1 = exprtk::integrate(expression, x, T(-1), T(1));

// Integrate in domain [-1,1] using name of x variable
T area2 = exprtk::integrate(expression, "x", T(-1), T(1));


(e) derivative
This free function will attempt to perform a numerical differentiation
of a single variable compiled expression at a given point for a  given
epsilon, using a  variant of Newton's  difference quotient called  the
five-point stencil method. The derivative function has two  overloads,
where  the  variable of  differentiation  can either  be  passed as  a
reference or as a name in string form. Example usage of the derivative
function is as follows:

typedef exprtk::parser<T>             parser\_t;
typedef exprtk::expression<T>     expression\_t;
typedef exprtk::symbol\_table<T> symbol\_table\_t;

std::string expression\_string = "sqrt(1 - (x\^2))";

T x = T(0);

symbol\_table\_t symbol\_table;
symbol\_table.add\_variable("x",x);

expression\_t expression;
expression.register\_symbol\_table(symbol\_table);

parser\_t parser;
parser.compile(expression\_string,expression);

....

// Differentiate expression at value of x = 12.3 using a reference
// to the x variable
x = T(12.3);
T derivative1 = exprtk::derivative(expression, x);

// Differentiate expression where value x = 45.6 using name
// of the x variable
x = T(45.6);
T derivative2 = exprtk::derivative(expression, "x");


(f) second\_derivative
This  free  function  will  attempt  to  perform  a  numerical  second
derivative of a single variable  compiled expression at a given  point
for a given epsilon, using  a variant of Newton's difference  quotient
method. The second\_derivative function  has  two overloads, where  the
variable of differentiation can either be passed as a reference or  as
a name in string form. Example usage of the second\_derivative function
is as follows:

typedef exprtk::parser<T>             parser\_t;
typedef exprtk::expression<T>     expression\_t;
typedef exprtk::symbol\_table<T> symbol\_table\_t;

std::string expression\_string = "sqrt(1 - (x\^2))";

T x = T(0);

symbol\_table\_t symbol\_table;
symbol\_table.add\_variable("x",x);

expression\_t expression;
expression.register\_symbol\_table(symbol\_table);

parser\_t parser;
parser.compile(expression\_string,expression);

....

// Second derivative of expression where value of x = 12.3 using a
// reference to x variable
x = T(12.3);
T derivative1 = exprtk::second\_derivative(expression,x);

// Second derivative of expression where value of x = 45.6 using
// name of x variable
x = T(45.6);
T derivative2 = exprtk::second\_derivative(expression, "x");


(g) third\_derivative
This  free  function  will  attempt  to  perform  a  numerical   third
derivative of a single variable  compiled expression at a given  point
for a given epsilon, using  a variant of Newton's difference  quotient
method. The  third\_derivative function  has two  overloads, where  the
variable of differentiation can either be passed as a reference or  as
a name in string form. Example  usage of the third\_derivative function
is as follows:

typedef exprtk::parser<T>             parser\_t;
typedef exprtk::expression<T>     expression\_t;
typedef exprtk::symbol\_table<T> symbol\_table\_t;

std::string expression\_string = "sqrt(1 - (x\^2))";

T x = T(0);

symbol\_table\_t symbol\_table;
symbol\_table.add\_variable("x",x);

expression\_t expression;
expression.register\_symbol\_table(symbol\_table);

parser\_t parser;
parser.compile(expression\_string,expression);

....

// Third derivative of expression where value of x = 12.3 using a
// reference to the x variable
x = T(12.3);
T derivative1 = exprtk::third\_derivative(expression, x);

// Third derivative of expression where value of x = 45.6 using
// name of the x variable
x = T(45.6);
T derivative2 = exprtk::third\_derivative(expression, "x");

~~~~~~~~~~~~~~~~~~~~~~~~~~~~~~~~~~~~~~~~~~~~~~~~~~~~~~~~~~

[SECTION 24 - BENCHMARKING]
As part of the ExprTk package there is an expression benchmark utility
named 'exprtk\_benchmark'. The utility attempts to determine expression
evaluation  speed (or  rate of  evaluations -  evals  per  second), by
evaluating each expression numerous times and mutating the  underlying
variables  of  the  expression between  each  evaluation.  The utility
assumes any  valid ExprTk  expression (containing  conditionals, loops
etc), however  it will  only make  use of  a predefined  set of scalar
variables, namely: a, b, c, x, y, z and w. That being said expressions
themselves  can  contain any  number  of local  variables,  vectors or
strings. There are two modes of operation:

(1) Default
(2) User Specified Expressions


(1) Default
The default mode is  enabled simply by executing  the exprtk\_benchmark
binary with no command line parameters. In this mode a predefined  set
of expressions will be evaluated in three phases:

(a) ExprTk evaluation
(b) Native evaluation
(c) ExprTk parse


In the first two  phases (a and b)  a list of predefined  (hard-coded)
expressions  will  be  evaluated using  both  ExprTk  and native  mode
implementations.  This  is  done so  as  to  compare evaluation  times
between ExprTk and native implementations. The set of expressions used
are as follows:

(01) (y + x)
(02) 2 * (y + x)
(03) (2 * y + 2 * x)
(04) ((1.23 * x\^2) / y) - 123.123
(05) (y + x / y) * (x - y / x)
(06) x / ((x + y) + (x - y)) / y
(07) 1 - ((x * y) + (y / x)) - 3
(08) (5.5 + x) + (2 * x - 2 / 3 * y) * (x / 3 + y / 4) + (y + 7.7)
(09) 1.1x\^1 + 2.2y\^2 - 3.3x\^3 + 4.4y\^15 - 5.5x\^23 + 6.6y\^55
(10) sin(2 * x) + cos(pi / y)
(11) 1 - sin(2 * x) + cos(pi / y)
(12) sqrt(111.111 - sin(2 * x) + cos(pi / y) / 333.333)
(13) (x\^2 / sin(2 * pi / y)) - x / 2
(14) x + (cos(y - sin(2 / x * pi)) - sin(x - cos(2 * y / pi))) - y
(15) clamp(-1.0, sin(2 * pi * x) + cos(y / 2 * pi), +1.0)
(16) max(3.33, min(sqrt(1 - sin(2 * x) + cos(pi / y) / 3), 1.11))
(17) if((y + (x * 2.2)) <= (x + y + 1.1), x - y, x*y) + 2 * pi / x


The  third  and  final  phase  (c),  is  used  to  determine   average
compilation rates  (compiles per  second) for  expressions of  varying
complexity. Each expression is compiled 100K times and the average for
each expression is output.


(2) User Specified Expressions
In this mode two parameters are passed to the utility via the  command
line:

(a) A name of a text file containing one expression per line
(b) An integer representing the number of evaluations per expression


An  example execution  of the  benchmark utility  in this  mode is  as
follows:

./exprtk\_benchmark my\_expressions.txt 1000000


The  above  invocation  will  load  the  expressions  from  the   file
'my\_expressions.txt' and will then proceed to evaluate each expression
one million  times, varying  the above  mentioned variables  (x, y,  z
etc.) between each evaluation, and at the end of each expression round
a print out of  running times, result of a single evaluation and total
sum of results is provided as demonstrated below:

Expression 1 of 7 4.770 ns 47700 ns  ( 9370368.0) '((((x+y)+z)))'
Expression 2 of 7 4.750 ns 47500 ns  ( 1123455.9) '((((x+y)-z)))'
Expression 3 of 7 4.766 ns 47659 ns  (21635410.7) '((((x+y)*z)))'
Expression 4 of 7 5.662 ns 56619 ns  ( 1272454.9) '((((x+y)/z)))'
Expression 5 of 7 4.950 ns 49500 ns  ( 4123455.9) '((((x-y)+z)))'
Expression 6 of 7 7.581 ns 75810 ns  (-4123455.9) '((((x-y)-z)))'
Expression 7 of 7 4.801 ns 48010 ns  (       0.0) '((((x-y)*z)))'


The benchmark utility can be very useful when investigating evaluation
efficiency  issues with  ExprTk or  simply during  the prototyping  of
expressions. As an example, lets take the following expression:

1 / sqrt(2x) * e\^(3y)


Lets say we would like to determine which sub-part  of the  expression
takes the  most time  to evaluate  and perhaps  attempt to  rework the
expression based on the results. In order to do this we will create  a
text file  called 'test.txt'  and then  proceed to  make some educated
guesses  about  how  to  break   the  expression  up  into  its   more
'interesting' sub-parts which we will  then add as one expression  per
line to the file. An example breakdown may be as follows:

1 / sqrt(2x) * e\^(3y)
1 / sqrt(2x)
e\^(3y)


The  benchmark with  the given  file, where  each expression  will be
evaluated 100K times can be executed as follows:

./exprtk\_benchmark test.txt 100000
Expr 1 of 3 90.340 ns 9034000 ns (296417859.3) '1/sqrt(2x)*e\^(3y)'
Expr 2 of 3 11.100 ns 1109999 ns (    44267.3) '1/sqrt(2x)'
Expr 3 of 3 77.830 ns 7783000 ns (615985286.6) 'e\^(3y)'
[*] Number Of Evals:         300000
[*] Total Time:              0.018sec
[*] Total Single Eval Time:  0.000ms


From the results above we conclude that the third expression  (e\^(3y))
consumes the largest amount of time. The variable 'e', as used in both
the  benchmark  and  in the  expression,  is  an approximation  of the
transcendental mathematical constant e (2.71828182845904...) hence the
sub-expression  should perhaps be  modified  to use the generally more
efficient built-in 'exp' function.

./exprtk\_benchmark test.txt 1000000
Expr 1 of 5 86.563 ns 8656300ns (296417859.6) '1/sqrt(2x)*e\^(3y)'
Expr 2 of 5 40.506 ns 4050600ns (296417859.6) '1/sqrt(2x)*exp(3y)'
Expr 3 of 5 14.248 ns 1424799ns (    44267.2) '1/sqrt(2x)'
Expr 4 of 5 88.840 ns 8884000ns (615985286.9) 'e\^(3y)'
Expr 5 of 5 29.267 ns 2926699ns (615985286.9) 'exp(3y)'
[*] Number Of Evals:        5000000
[*] Total Time:             0.260sec
[*] Total Single Eval Time: 0.000ms


The above output demonstrates the  results from  making the previously
mentioned modification to the expression. As can be seen the new  form
of the expression using the 'exp' function reduces the evaluation time
by over 50\%, in other words increases the evaluation rate by two fold.

~~~~~~~~~~~~~~~~~~~~~~~~~~~~~~~~~~~~~~~~~~~~~~~~~~~~~~~~~~

[SECTION 25 - EXPRTK NOTES]
The following is a list of facts and suggestions one may want to take
into account when using ExprTk:

(00) Precision  and performance  of expression  evaluations are  the
dominant principles of the ExprTk library.

(01) ExprTk  uses a  rudimentary imperative  programming model  with
syntax based  on languages  such as  Pascal and  C. Furthermore
ExprTk  is  an LL(2)  type  grammar and  is  processed using  a
recursive descent parsing algorithm.

(02) Supported types are float, double, long double and MPFR/GMP.

(03) Standard mathematical operator precedence is applied (BEDMAS).

(04) Results of expressions that are deemed as being 'valid' are  to
exist within the set of Real numbers. All other results will be
of the value: Not-A-Number (NaN).

(05) Supported   user  defined   types   are   numeric  and   string
variables, numeric vectors and functions.

(06) All  reserved words,  keywords, variable,  vector, string   and
function names are case-insensitive.

(07) Variable, vector, string variable and function names must begin
with  a letter  (A-Z or  a-z), then  can be  comprised of  any
combination of letters, digits,  underscores and dots. (eg:  x,
var1 or power\_func99, person.age, item.size.0)

(08) Expression lengths and sub-expression lists are limited only by
storage capacity.

(09) The  life-time of  objects registered  with or  created from  a
specific symbol-table must span  at least the life-time  of the
compiled expressions which utilise objects, such as  variables,
of that  symbol-table, otherwise  the result  will be undefined
behavior.

(10) Equal  and  Nequal  are  normalised-epsilon  equality routines,
which use epsilons of 0.0000000001 and 0.000001 for double  and
float types respectively.

(11) All trigonometric functions  assume radian input  unless stated
otherwise.

(12) Expressions may contain  white-space characters such  as space,
tabs, new-lines, control-feed et al.
('\\n', '\\r', '\\t', '\\b', '\\v', '\\f')

(13) Strings may be comprised of any combination of letters, digits
special characters including (~!@\#\$\%\^\&*()[]|=+ ,./?<>;:"`~\_) or
hexadecimal escaped sequences (eg: \\0x30) and must be enclosed
with single-quotes.
eg: 'Frankly my dear, \\0x49 do n0t give a damn!'

(14) User defined  normal functions  can have  up to  20 parameters,
where as  user defined  generic-functions and  vararg-functions
can have an unlimited number of parameters.

(15) The inbuilt polynomial functions can be at most of degree 12.

(16) Where appropriate constant folding optimisations may be applied.
(eg: The expression '2 + (3 - (x / y))' becomes '5 - (x / y)')

(17) If the strength reduction compilation option has been enabled,
then where applicable strength reduction optimisations may be
applied.

(18) String  processing capabilities  are available  by default.  To
turn them  off, the  following needs  to be  defined at compile
time: exprtk\_disable\_string\_capabilities

(19) Composited functions can call themselves or any other functions
that have been defined prior to their own definition.

(20) Recursive calls made from within composited functions will have
a stack size bound by the stack of the executing architecture.

(21) User  defined functions  by default  are assumed  to have  side
effects. As such an "all constant parameter" invocation of such
functions wont result in constant folding. If the function  has
no side effects then that  can be noted during the  constructor
of  the  ifunction  allowing it  to  be  constant folded  where
appropriate.

(22) The entity relationship between symbol\_table and an  expression
is many-to-many. However the intended 'typical' use-case  where
possible, is to have a single symbol table manage the  variable
and function requirements of multiple expressions.

(23) The common use-case  for an expression  is to have  it compiled
only  ONCE  and  then subsequently  have it  evaluated multiple
times. An extremely  inefficient and suboptimal  approach would
be to recompile an expression  from its string form every  time
it requires evaluating.

(24) It is  strongly recommended  that  the  return value  of method
invocations from  the parser  and symbol\_table  types be  taken
into account. Specifically the  'compile' method of the  parser
and the 'add\_xxx'  set of methods  of the symbol\_table  as they
denote either the success or failure state of the invoked call.
Continued processing from  a failed  state without having first
rectified the underlying issue  will in turn result  in further
failures and undefined behaviours.

(25) The following are examples of compliant floating point value
representations:

(1) 12345         (5) -123.456
(2) +123.456e+12  (6) 123.456E-12
(3) +012.045e+07  (7) .1234
(4) 123.456f      (8) -321.654E+3L

(26) Expressions may contain any of the following comment styles:

(1) // .... \\n
(2) \#  .... \\n
(3) /* .... */

(27) The  'null'  value  type  is  a  special  non-zero  type   that
incorporates specific semantics when undergoing operations with
the standard numeric type. The following is a list of type  and
boolean results associated with the use of 'null':

(1) null  +,-,*,/,\%  x    --> x
(2) x     +,-,*,/,\%  null --> x
(3) null  +,-,*,/,\%  null --> null
(4) null     ==      null --> true
(5) null     ==      x    --> true
(6) x        ==      null --> true
(7) x        !=      null --> false
(8) null     !=      null --> false
(9) null     !=      x    --> false

(28) The following is a list  of reserved words and symbols  used by
ExprTk. Attempting to  add a variable  or custom function  to a
symbol table using any of  the reserved words will result  in a
failure.

abs, acos, acosh, and, asin, asinh, atan, atan2, atanh, avg,
break, case,  ceil, clamp,  continue, cosh,  cos, cot,  csc,
default,  deg2grad, deg2rad,  else, equal,  erfc, erf,  exp,
expm1, false, floor, for, frac, grad2deg, hypot, iclamp, if,
ilike, in, inrange, in, like, log, log10, log1p, log2, logn,
mand,  max,  min,  mod,  mor,  mul,  nand,  ncdf,  nor, not,
not\_equal,  not,  null, or,  pow,  rad2deg, repeat,  return,
root, roundn, round,  sec, sgn, shl,  shr, sinc, sinh,  sin,
sqrt, sum, swap, switch, tanh, tan, true, trunc, until, var,
while, xnor, xor, xor

(29) Every valid ExprTk statement is a "value returning" expression.
Unlike some languages that limit the types of expressions  that
can be  performed in  certain situations,  in ExprTk  any valid
expression can be used in any "value consuming" context. eg:

var y := 3;
for (var x := switch
{
	case 1  :  7;
	case 2  : -1 + ~{var x{};};
	default :  y > 2 ? 3 : 4;
};
x != while (y > 0) { y -= 1; };
x -= {
	if (min(x,y) < 2 * max(x,y))
	x + 2;
	else
	x + y - 3;
}
)
{
	(x + y) / (x - y);
}

(30) It is recommended when prototyping expressions that the  ExprTk
REPL be utilised, as it supports all the features available  in
the library,  including complete  error analysis,  benchmarking
and   dependency   dumps    etc   which   allows    for   rapid
coding/prototyping  and  debug  cycles  without  the  hassle of
having to  recompile test  programs with  expressions that have
been hard-coded. It's also a  good source of truth for  how the
library's various features can be applied.

(31) For performance considerations,  one should assume  the actions
of expression, symbol  table and parser  instance instantiation
and destruction, and the expression compilation process  itself
to be of high latency. Hence none of them should be part of any
performance  critical  code  paths, and  should  instead  occur
entirely either before or after such code paths.

(32) Deep  copying  an  expression  instance  for  the  purposes  of
persisting to disk or otherwise transmitting elsewhere with the
intent to 'resurrect' the  expression instance later on  is not
possible due  to the  reasons described  in the  final note  of
Section 10. The recommendation is to instead simply persist the
string form  of the  expression and  compile the  expression at
run-time on the target.

(33) Before jumping in and using ExprTk, do take the time to  peruse
the documentation and all of the examples, both in the main and
the extras  distributions. Having  an informed  general view of
what can and  can't be done,  and how something  should be done
with ExprTk, will  likely result in  a far more  productive and
enjoyable programming experience.

~~~~~~~~~~~~~~~~~~~~~~~~~~~~~~~~~~~~~~~~~~~~~~~~~~~~~~~~~~

[SECTION 26 - SIMPLE EXPRTK EXAMPLE]
The following is a  simple yet complete example  demonstrating typical
usage of the ExprTk Library.  The example instantiates a symbol  table
object, adding to it  three variables named x,  y and z, and  a custom
user defined function, that accepts only two parameters, named myfunc.
The  example then  proceeds to  instantiate an  expression object  and
register to it the symbol table instance.

A parser is  then instantiated, and  the string representation  of the
expression  and  the  expression object  are  passed  to the  parser's
compile  method   for  compilation.   If  an   error  occurred  during
compilation, the compile method will return false, leading to a series
of  error diagnostics  being printed  to stdout.  Otherwise the  newly
compiled expression is evaluated  by invoking  the expression object's
value method, and subsequently printing the result  of the computation
to stdout.


--- snip ---
\#include <cstdio>
\#include <string>

\#include "exprtk.hpp"

template <typename T>
struct myfunc : public exprtk::ifunction<T>
{
	myfunc() : exprtk::ifunction<T>(2) {}
	
	T operator()(const T\& v1, const T\& v2)
	{
		return T(1) + (v1 * v2) / T(3);
	}
};

int main()
{
	typedef exprtk::symbol\_table<double> symbol\_table\_t;
	typedef exprtk::expression<double>     expression\_t;
	typedef exprtk::parser<double>             parser\_t;
	typedef exprtk::parser\_error::type          error\_t;
	
	std::string expression\_str =
	"z := 2 myfunc([4 + sin(x / pi)\^3],y \^ 2)";
	
	double x = 1.1;
	double y = 2.2;
	double z = 3.3;
	
	myfunc<double> mf;
	
	symbol\_table\_t symbol\_table;
	symbol\_table.add\_constants();
	symbol\_table.add\_variable("x",x);
	symbol\_table.add\_variable("y",y);
	symbol\_table.add\_variable("z",z);
	symbol\_table.add\_function("myfunc",mf);
	
	expression\_t expression;
	expression.register\_symbol\_table(symbol\_table);
	
	parser\_t parser;
	
	if (!parser.compile(expression\_str,expression))
	{
		// A compilation error has occurred. Attempt to
		// print all errors to stdout.
		
		printf("Error: \%s\\tExpression: \%s\\n",
		parser.error().c\_str(),
		expression\_str.c\_str());
		
		for (std::size\_t i = 0; i < parser.error\_count(); ++i)
		{
			// Include the specific nature of each error
			// and its position in the expression string.
			
			error\_t error = parser.get\_error(i);
			
			printf("Error: \%02d Position: \%02d "
			"Type: [\%s] "
			"Message: \%s "
			"Expression: \%s\\n",
			static\_cast<int>(i),
			static\_cast<int>(error.token.position),
			exprtk::parser\_error::to\_str(error.mode).c\_str(),
			error.diagnostic.c\_str(),
			expression\_str.c\_str());
		}
		
		return 1;
	}
	
	// Evaluate the expression and obtain its result.
	
	double result = expression.value();
	
	printf("Result: \%10.5f\\n",result);
	
	return 0;
}
--- snip ---

~~~~~~~~~~~~~~~~~~~~~~~~~~~~~~~~~~~~~~~~~~~~~~~~~~~~~~~~~~

[SECTION 27 - BUILD OPTIONS]
When building ExprTk there are a number of defines that will enable or
disable certain features and  capabilities. The defines can  either be
part of a compiler command line switch or scoped around the include to
the ExprTk header. The defines are as follows:

(01) exprtk\_enable\_debugging
(02) exprtk\_disable\_comments
(03) exprtk\_disable\_break\_continue
(04) exprtk\_disable\_sc\_andor
(05) exprtk\_disable\_return\_statement
(06) exprtk\_disable\_enhanced\_features
(07) exprtk\_disable\_string\_capabilities
(08) exprtk\_disable\_superscalar\_unroll
(09) exprtk\_disable\_rtl\_io\_file
(10) exprtk\_disable\_rtl\_vecops
(11) exprtk\_disable\_caseinsensitivity


(01) exprtk\_enable\_debugging
This define will enable printing of debug information to stdout during
the compilation process.

(02) exprtk\_disable\_comments
This define will disable the ability for expressions to have comments.
Expressions that have comments when parsed with a build that has  this
option, will result in a compilation failure.

(03) exprtk\_disable\_break\_continue
This  define  will  disable  the  loop-wise  'break'  and   'continue'
capabilities. Any expression that contains those keywords will  result
in a compilation failure.

(04) exprtk\_disable\_sc\_andor
This define  will disable  the short-circuit  '\&' (and)  and '|'  (or)
operators

(05) exprtk\_disable\_return\_statement
This define will disable use of return statements within expressions.

(06) exprtk\_disable\_enhanced\_features
This  define  will  disable all  enhanced  features  such as  strength
reduction and special  function optimisations and  expression specific
type instantiations.  This feature  will reduce  compilation times and
binary sizes but will  also result in massive  performance degradation
of expression evaluations.

(07) exprtk\_disable\_string\_capabilities
This  define  will  disable all  string  processing  capabilities. Any
expression that contains a string or string related syntax will result
in a compilation failure.

(08) exprtk\_disable\_superscalar\_unroll
This define will set  the loop unroll batch  size to 4 operations  per
loop  instead of  the default  8 operations.  This define  is used  in
operations that  involve vectors  and aggregations  over vectors. When
targeting  non-superscalar  architectures, it  may  be recommended  to
build using this particular option if efficiency of evaluations is  of
concern.

(09) exprtk\_disable\_rtl\_io\_file
This  define will  disable  the  file I/O  RTL package  features. When
present, any  attempts to register  the file I/O package with  a given
symbol table will fail causing a compilation error.

(10) exprtk\_disable\_rtl\_vecops
This define will  disable the extended  vector operations RTL  package
features. When present, any attempts to register the vector operations
package with  a given  symbol table  will fail  causing a  compilation
error.

(11) exprtk\_disable\_caseinsensitivity
This define  will disable  case-insensitivity when  matching variables
and  functions. Furthermore  all reserved  and keywords  will only  be
acknowledged when in all lower-case.

~~~~~~~~~~~~~~~~~~~~~~~~~~~~~~~~~~~~~~~~~~~~~~~~~~~~~~~~~~

[SECTION 28 - FILES]
The source distribution of ExprTk is comprised of the following set of
files:

(00) Makefile
(01) readme.txt
(02) exprtk.hpp
(03) exprtk\_test.cpp
(04) exprtk\_benchmark.cpp
(05) exprtk\_simple\_example\_01.cpp
(06) exprtk\_simple\_example\_02.cpp
(07) exprtk\_simple\_example\_03.cpp
(08) exprtk\_simple\_example\_04.cpp
(09) exprtk\_simple\_example\_05.cpp
(10) exprtk\_simple\_example\_06.cpp
(11) exprtk\_simple\_example\_07.cpp
(12) exprtk\_simple\_example\_08.cpp
(13) exprtk\_simple\_example\_09.cpp
(14) exprtk\_simple\_example\_10.cpp
(15) exprtk\_simple\_example\_11.cpp
(16) exprtk\_simple\_example\_12.cpp
(17) exprtk\_simple\_example\_13.cpp
(18) exprtk\_simple\_example\_14.cpp
(19) exprtk\_simple\_example\_15.cpp
(20) exprtk\_simple\_example\_16.cpp
(21) exprtk\_simple\_example\_17.cpp
(22) exprtk\_simple\_example\_18.cpp
(23) exprtk\_simple\_example\_19.cpp

~~~~~~~~~~~~~~~~~~~~~~~~~~~~~~~~~~~~~~~~~~~~~~~~~~~~~~~~~~

[SECTION 29 - LANGUAGE STRUCTURE]
The following  are the  various language  structures available  within
ExprTk and their structural representations.

(00) If Statement
(01) Else Statement
(02) Ternary Statement
(03) While Loop
(04) Repeat Until Loop
(05) For Loop
(06) Switch Statement
(07) Multi Subexpression Statement
(08) Multi Case-Consequent Statement
(09) Variable Definition Statement
(10) Vector Definition Statement
(11) String Definition Statement
(12) Range Statement
(13) Return Statement


(00) - If Statement
+-------------------------------------------------------------+
|                                                             |
|   [if] ---> [(] ---> [condition] -+-> [,] -+                |
|                                   |        |                |
|   +---------------<---------------+        |                |
|   |                                        |                |
|   |  +------------------<------------------+                |
|   |  |                                                      |
|   |  +--> [consequent] ---> [,] ---> [alternative] ---> [)] |
|   |                                                         |
|   +--> [)] --+-> [{] ---> [expression*] ---> [}] --+        |
|              |                                     |        |
|              |                +---------<----------+        |
|   +----<-----+                |                             |
|   |                           v                             |
|   +--> [consequent] --> [;] -{*}-> [else-statement]         |
|                                                             |
+-------------------------------------------------------------+


(01) - Else Statement
+-------------------------------------------------------------+
|                                                             |
|   [else] -+-> [alternative] ---> [;]                        |
|           |                                                 |
|           +--> [{] ---> [expression*] ---> [}]              |
|           |                                                 |
|           +--> [if-statement]                               |
|                                                             |
+-------------------------------------------------------------+


(02) - Ternary Statement
+-------------------------------------------------------------+
|                                                             |
|   [condition] ---> [?] ---> [consequent] ---> [:] --+       |
|                                                     |       |
|   +------------------------<------------------------+       |
|   |                                                         |
|   +--> [alternative] --> [;]                                |
|                                                             |
+-------------------------------------------------------------+


(03) - While Loop
+-------------------------------------------------------------+
|                                                             |
|   [while] ---> [(] ---> [condition] ---> [)] ---+           |
|                                                 |           |
|   +----------------------<----------------------+           |
|   |                                                         |
|   +--> [{] ---> [expression*] ---> [}]                      |
|                                                             |
+-------------------------------------------------------------+


(04) - Repeat Until Loop
+-------------------------------------------------------------+
|                                                             |
|   [repeat] ---> [expression*] ---+                          |
|                                  |                          |
|   +--------------<---------------+                          |
|   |                                                         |
|   +--> [until] ---> [(] ---> [condition] --->[)]            |
|                                                             |
+-------------------------------------------------------------+


(05) - For Loop
+-------------------------------------------------------------+
|                                                             |
|   [for] ---> [(] -+-> [initialise expression] --+--+        |
|                   |                             |  |        |
|                   +------------->---------------+  v        |
|                                                    |        |
|   +-----------------------<------------------------+        |
|   |                                                         |
|   +--> [;] -+-> [condition] -+-> [;] ---+                   |
|             |                |          |                   |
|             +------->--------+          v                   |
|                                         |                   |
|   +------------------<---------+--------+                   |
|   |                            |                            |
|   +--> [increment expression] -+-> [)] --+                  |
|                                          |                  |
|   +------------------<-------------------+                  |
|   |                                                         |
|   +--> [{] ---> [expression*] ---> [}]                      |
|                                                             |
+-------------------------------------------------------------+


(06) - Switch Statement
+-------------------------------------------------------------+
|                                                             |
|   [switch] ---> [{] ---+                                    |
	|                        |                                    |
	|   +---------<----------+-----------<-----------+            |
	|   |                                            |            |
	|   +--> [case] ---> [condition] ---> [:] ---+   |            |
	|                                            |   |            |
	|   +-------------------<--------------------+   |            |
	|   |                                            |            |
	|   +--> [consequent] ---> [;] --------->--------+            |
	|   |                                            |            |
	|   |                                            |            |
	|   +--> [default] ---> [consequent] ---> [;] ---+            |
	|   |                                            |            |
	|   +---------------------<----------------------+            |
	|   |                                                         |
	|   +--> [}]                                                  |
|                                                             |
+-------------------------------------------------------------+


(07) - Multi Subexpression Statement
+-------------------------------------------------------------+
|                                                             |
|                   +--------------<---------------+          |
|                   |                              |          |
|   [~] ---> [{\\(] -+-> [expression] -+-> [;\\,] ---+          |
	|                                     |                       |
	|   +----------------<----------------+                       |
	|   |                                                         |
	|   +--> [}\\)]                                                |
|                                                             |
+-------------------------------------------------------------+


(08) - Multi Case-Consequent Statement
+-------------------------------------------------------------+
|                                                             |
|   [[*]] ---> [{] ---+                                       |
	|                     |                                       |
	|   +--------<--------+--------------<----------+             |
	|   |                                           |             |
	|   +--> [case] ---> [condition] ---> [:] ---+  |             |
	|                                            |  |             |
	|   +-------------------<--------------------+  |             |
	|   |                                           |             |
	|   +--> [consequent] ---> [;] ---+------>------+             |
	|                                 |                           |
	|                                 +--> [}]                    |
|                                                             |
+-------------------------------------------------------------+


(09) - Variable Definition Statement
+-------------------------------------------------------------+
|                                                             |
|   [var] ---> [symbol] -+-> [:=] -+-> [expression] -+-> [;]  |
|                        |         |                 |        |
|                        |         +-----> [{}] -->--+        |
|                        |                           |        |
|                        +------------->-------------+        |
|                                                             |
+-------------------------------------------------------------+


(10) - Vector Definition Statement
+-------------------------------------------------------------+
|                                                             |
|   [var] ---> [symbol] ---> [[] ---> [constant] ---> []] --+ |
|                                                           | |
|   +---------------------------<---------------------------+ |
|   |                                                         |
|   |                   +--------->---------+                 |
|   |                   |                   |                 |
|   +--> [:=] ---> [{] -+-+-> [expression] -+-> [}] ---> [;]  |
|                         |                 |                 |
|                         +--<--- [,] <-----+                 |
|                                                             |
+-------------------------------------------------------------+


(11) - String Definition Statement
+-------------------------------------------------------------+
|                                                             |
|   [var] --> [symbol] --> [:=] --> [str-expression] ---> [;] |
|                                                             |
+-------------------------------------------------------------+


(12) - Range Statement
+-------------------------------------------------------------+
|                                                             |
|      +-------->--------+                                    |
|      |                 |                                    |
| [[] -+-> [expression] -+-> [:] -+-> [expression] -+--> []]  |
|                                 |                 |         |
|                                 +-------->--------+         |
|                                                             |
+-------------------------------------------------------------+


(13) - Return Statement
+-------------------------------------------------------------+
|                                                             |
|   [return] ---> [[] -+-> [expression] -+-> []] ---> [;]     |
|                      |                 |                    |
|                      +--<--- [,] <-----+                    |
|                                                             |
+-------------------------------------------------------------+
	
